\documentclass[
	% -- opções da classe memoir --
	12pt,				% tamanho da fonte
	a4paper,			% tamanho do papel.
	oneside,
	% -- opções do pacote babel --
	english			% idioma adicional para hifenização, o último idioma é o principal do documento
	]{abntex2}

% ---
% Pacotes básicos
% ---
\usepackage{lmodern}			% Usa a fonte Latin Modern
\usepackage[T1]{fontenc}		% Selecao de codigos de fonte.
\usepackage[utf8]{inputenc}		% Codificacao do documento (conversão automática dos acentos)
\usepackage{indentfirst}		% Indenta o primeiro parágrafo de cada seção.
\usepackage{color}				% Controle das cores
\usepackage{graphicx}			% Inclusão de gráficos
\usepackage{microtype} 			% para melhorias de justificação
% ---

\usepackage{url}

% ---
% Pacotes de citações
% ---
\usepackage[brazilian,hyperpageref]{backref}	 % Paginas com as citações na bibl
\usepackage[alf]{abntex2cite}	% Citações padrão ABNT

\usepackage[color]{coqdoc}
\usepackage{cspsymb}
\usepackage{amsmath}

% Proof trees in the style of the sequent calculus
\usepackage{bussproofs}

% Unnumbered theorem-like environments
\usepackage{amsthm}

\theoremstyle{definition}
\newtheorem*{definition}{Definition}

%%%%% Coq Proof Script Visualiser %%%%%

\usepackage{tabularx}
\usepackage[mathletters]{ucs}
%\usepackage[utf8x]{inputenc}
%\usepackage{unicode}
%\usepackage{fullpage}
\usepackage{multirow}
\usepackage{longtable, array}

\usepackage{textcomp} % required to use \textquotesingle

% The width of the coq step column
\providecommand{\stepcolwidth}{0.2}
% The width of the frac rule in coq style output
\providecommand{\rulewidth}{1.0}
% The frac rule thickness in coq style output
\providecommand{\rulethickness}{0.2}
% The margin strech for the table cells
\renewcommand{\arraystretch}{1.3}
% The horizontal alignment of proof situations
\providecommand{\coqpsvpsitalign}{\centering}
% The command for printing the frac rule in coq style output
\providecommand{\fracrule}{\rule{\rulewidth\hsize}{\rulethickness mm}}
% The command for generating a proof situation environment
\providecommand{\coqpsvpsit}[1]{\begin{minipage}[t]{0.7\textwidth}\coqpsvpsitalign#1\end{minipage}}

% Now we define the table headers
\providecommand{\coqpsvstephdr}{Next step in Coq}
\providecommand{\coqpsvsithdr}{\multicolumn{1}{>{\centering\arraybackslash}c|}{Proof situation}}

\newcolumntype{S}{m{\stepcolwidth\textwidth}}
\newcolumntype{P}{>{\coqpsvpsitalign\arraybackslash}p{0.700000\textwidth}}

%%%%%%%%%%%%%%%%%%%%%%%%%%%%%%%%%%%%%%%

% ---
% CONFIGURAÇÕES DE PACOTES
% ---

% ---
% Configurações do pacote backref
% Usado sem a opção hyperpageref de backref
\renewcommand{\backrefpagesname}{Citado na(s) página(s):~}
% Texto padrão antes do número das páginas
\renewcommand{\backref}{}
% Define os textos da citação
\renewcommand*{\backrefalt}[4]{
	\ifcase #1 %
		Nenhuma citação no texto.%
	\or
		Citado na página #2.%
	\else
		Citado #1 vezes nas páginas #2.%
	\fi}%
% ---

% ---
% Informações de dados para CAPA e FOLHA DE ROSTO
% ---
\titulo{A Theory for Communicating\\ Sequential Processes in Coq}
\autor{Carlos Alberto da Silva Carvalho de Freitas}
\local{Recife}
\data{2020}
\orientador{Gustavo Henrique Porto de Carvalho}
\instituicao{%
  Universidade Federal de Pernambuco
  \par
  Centro de Informática
  \par
  Bachelor in Computer Engineering}
\tipotrabalho{Monografia (Bacharelado)}
% O preambulo deve conter o tipo do trabalho, o objetivo,
% o nome da instituição e a área de concentração
\preambulo{
	A B.Sc. Dissertation presented to the Centro de Informática of
	Universidade Federal de Pernambuco in partial fulfillment of the requirements
	for the degree of Bachelor in Computer Engineering.
}
% ---


% ---
% Configurações de aparência do PDF final

% alterando o aspecto da cor azul
\definecolor{blue}{RGB}{41,5,195}

% informações do PDF
\makeatletter
\hypersetup{
     	%pagebackref=true,
		pdftitle={\@title},
		pdfauthor={\@author},
    	pdfsubject={\imprimirpreambulo},
	    pdfcreator={LaTeX with abnTeX2},
		pdfkeywords={abnt}{latex}{abntex}{abntex2}{trabalho acadêmico},
		colorlinks=true,       		% false: boxed links; true: colored links
    	linkcolor=blue,          	% color of internal links
    	citecolor=blue,        		% color of links to bibliography
    	filecolor=magenta,      		% color of file links
		urlcolor=blue,
		bookmarksdepth=4
}
\makeatother
% ---

% ---
% Posiciona figuras e tabelas no topo da página quando adicionadas sozinhas
% em um página em branco. Ver https://github.com/abntex/abntex2/issues/170
\makeatletter
\setlength{\@fptop}{5pt} % Set distance from top of page to first float
\makeatother
% ---

% ---
% Espaçamentos entre linhas e parágrafos
% ---

% O tamanho do parágrafo é dado por:
\setlength{\parindent}{1.3cm}

% Controle do espaçamento entre um parágrafo e outro:
\setlength{\parskip}{0.2cm}  % tente também \onelineskip

% ---
% compila o indice
% ---
\makeindex
% ---

% Definindo o hífen com um novo caractere matemático para corrigir espaçamento e tamanho do símbolo no "env math".
\mathchardef\mhyphen="2D

% Defining symbol for the CSP_Coq language.
\newcommand{\CSPcoq}{CSP\textsubscript{\textit{Coq}}}

% ----
% Início do documento
% ----
\begin{document}

% Seleciona o idioma do documento (conforme pacotes do babel)
\selectlanguage{english}

% Retira espaço extra obsoleto entre as frases.
\frenchspacing

% ----------------------------------------------------------
% ELEMENTOS PRÉ-TEXTUAIS
% ----------------------------------------------------------

% ---
% Capa
% ---
\imprimircapa
% ---

% ---
% Folha de rosto
% ---
\imprimirfolhaderosto
% ---

% ---
% Agradecimentos
% ---
\begin{agradecimentos}
This work would not have been possible without the teachings and guidance of my supervisor, Professor Gustavo Carvalho, whom I thank especially for the understanding against the adverse conditions in which this work was developed (due to a pandemic without precedence in contemporary history). I extent my thanks also to the other professors at the Centro de Informática of Universidade Federal de Pernambuco, who value quality education and encourage students to seek knowledge on a daily basis.

I cannot fully express my gratitude to my parents, Píndaro and Rozilda. All I am and everything I have achieved to this day I owe to them. The humble yet challenging wish of my late father, when I was only a few years old, for me to grow into a good person and make contributions to the society, has always resonated deeply with me. My mother's relentless encouragement of my education and the pride she bears in her voice when talking about me, inspires me everyday to be the best version of myself.

Finally, I want to give a special thanks to the people with whom I shared closely my emotions during the entire graduation course -- my classmates and my girlfriend, Nínive. The friendship and companionship of these people not only renewed my strength day after day, but made this journey enjoyable in a way that I could never have foreseen. Knowing that I could count on them made a huge difference, and it was yet another good reason to get up every morning.
\end{agradecimentos}
% ---

% ---
% RESUMOS
% ---

% resumo em inglês
\setlength{\absparsep}{18pt} % ajusta o espaçamento dos parágrafos do resumo
\begin{resumo}
 Theories of concurrency such as Communicating Sequential Processes (CSP) allow system specifications to be expressed clearly and analysed with precision. CSP specifications are typically analysed with the aid of FDR, a refinement checker (model checker) for CSP. However, the state explosion problem, common to model checkers in general, is a real constraint when attempting to verify system properties for large systems. An alternative is to check these properties via proof development. This work provides an initial formalisation of CSP in the Coq proof assistant, and evaluates how this theory compares to other theorem prover-based frameworks for the process algebra CSP. We develop an infrastructure for declaring syntactically and semantically correct CSP specifications in Coq, along with native support for process representation through Labelled Transition Systems (LTSs), in addition to traces refinement analysis with the support of the QuickChick tool.

 \textbf{Keywords}: CSP. Coq. Process algebra. Proof assistant. Traces refinement. QuickChick.
\end{resumo}

% resumo em português
\begin{resumo}[Resumo]
 \begin{otherlanguage*}{brazil}
   Teorias de concorrência, tais como \emph{Communicating Sequential Processes} (CSP), permitem que especificações de sistemas sejam descritas com clareza e analisadas com precisão. Especificações em CSP são normalmente analisadas em FDR, um verificador de refinamentos (verificador de modelos) para CSP. No entanto, o problema da explosão de estados, comum a verificadores de modelo em geral, é uma limitação real na tentativa de verificar propriedades de um sistema complexo. Uma alternativa é avaliar essas propriedades através do desenvolvimento de provas. Este trabalho fornece uma primeira formalização de CSP no assistente de provas Coq, além de compará-la com outros \emph{frameworks} baseados em provadores de teoremas para a álgebra de processos CSP. Portanto, criou-se uma infraestrutura para declarar especificações sintatica e semanticamente corretas de CSP em Coq, juntamente com um suporte nativo para a representação de processos por meio de sistemas de transições rotuladas (LTSs), além de análise de refinamento no modelo de \emph{traces} a partir do uso da ferramenta QuickChick.

   \textbf{Palavras-chave}: CSP. Coq. Álgebra de processos. Assistentes de provas. Refinamento no modelo de \emph{traces}. QuickChick.
 \end{otherlanguage*}
\end{resumo}

% ---
% inserir lista de ilustrações
% ---
\pdfbookmark[0]{\listfigurename}{lof}
\listoffigures*
\cleardoublepage
% ---

% ---
% inserir lista de tabelas
% ---
\pdfbookmark[0]{\listtablename}{lot}
\listoftables*
\cleardoublepage
% ---

% ---
% inserir lista de abreviaturas e siglas
% ---
\begin{siglas}
	\item[CSP] Communicating Sequential Processes
	\item[FDR] Failures-Divergence Refinement
	\item[LTS] Labelled Transition System
	\item[SOS] Structured Operational Semantics
	\item[UTP] Unified Theories of Programming
\end{siglas}
% ---

% ---
% inserir o sumario
% ---
\pdfbookmark[0]{\contentsname}{toc}
\tableofcontents*
\cleardoublepage
% ---


% ----------------------------------------------------------
% ELEMENTOS TEXTUAIS
% ----------------------------------------------------------
\textual

% ----------------------------------------------------------
% Chapter 1
% ----------------------------------------------------------
\chapter{Introduction}
\label{chapter:introduction}
% ----------------------------------------------------------

Concurrency is an attribute of any system that allows multiple components to perform operations at the same time. The understanding of this property is essential in modern programming because major areas, such as distributed and real-time systems, rely on this concept to work properly. As a result, the variety of applications enabled by the concurrency feature is broad: aircraft and industrial control systems, routing algorithms, peer-to-peer networks, client-server applications and parallel computation, to name a few.

Since concurrent systems may have parts that execute in parallel, the combination of ways in which these parts can interact raises the complexity in designing such systems. Phenomena like deadlock, livelock, nondeterminism and race condition can emerge from these interactions, so these issues must be addressed in order to avoid undesired behaviour. Typically, testing cannot provide enough evidence to guarantee properties such as deadlock freedom, divergence freedom and determinism for a given system.

That being said, CSP, a theory for Communicating Sequential Processes, introduces a convenient notation that allows concurrent systems to be described in a clear and accurate way. More than that, it has an underlying theory that enables designs to be analysed and proven correct with respect to desired properties. Several tools for analysing designs in CSP notation have been developed over the years including PAT, ARC and FDR. The FDR (Failures-Divergence Refinement) tool is the most popular refinement (model) checker for CSP, and it is responsible for making this process algebra a practical tool for specification, analysis and verification of systems. System analysis via FDR is achieved by allowing the user to make assertions about processes and then exploring every possible behaviour, if necessary, to check the truthfulness of the assertions made.

Although it is undeniable that FDR is a useful tool in the analysis of systems described in CSP, it has a limitation that is common to standard model checkers in general: the state explosion problem. An alternative way for deciding whether a system meets its specification is by proof development. Examples of this different approach are CSP-Prover~\cite{Roggenbach:CSP-Prover} and Isabelle/UTP~\cite{Woodcock:Isabelle/UTP}, both frameworks based on the theorem prover Isabelle. Nevertheless, to the best of our knowledge, there is not a theory for CSP in the Coq proof assistant yet \cite{bertot:coq}. Considering that, the main research question of this work is the following: how could we develop a theory for CSP in Coq, exploiting the main advantages of this proof assistant?

% ---
\section{Objectives}
% ---

The main objective (MO) of this work is to define in Coq a theory for concurrent systems, based on a limited scope of the process algebra CSP. This objective is unfolded into the following specific objectives (SO):

\begin{itemize}
	\item SO1: study CSP and frameworks based on this process algebra.
	\item SO2: define a syntax for CSP in Coq, based on a restricted version of the \CSPM{} language (machine readable language for CSP).
	\item SO3: provide support for the LTS-based (Labelled Transition System) representation, considering the Structured Operational Semantics (SOS) of CSP.
	\item SO4: make use of the QuickChick \cite{pierce:quickchick} tool to search for counterexamples of the traces refinement relation.
\end{itemize}

% ---
\section{An overview of \CSPcoq{}}
% ---

To illustrate \CSPcoq{}, our dialect of CSP in Coq, consider the following CSP process, which has been adapted from \citeonline[p.~32, example 2.3]{schneider1999}. This process represents a cloakroom attendant that might help a customer off or on with his coat, storing and retrieving coats as appropriate.
%
\begin{align}
	\mathit{channel\ coat\_on, coat\_off, store, retrieve, request\_coat, eat}\notag
\end{align}
\begin{align}
	\mathit{SYSTEM} =\ &\mathit{coat\_off \mathbin{{-}{>}} store \mathbin{{-}{>}} request\_coat \mathbin{{-}{>}} retrieve \mathbin{{-}{>}} coat\_on \mathbin{{-}{>}} SKIP}\notag\\
			 &[|\ \{\mathit{coat\_off, request\_coat, coat\_on}\}\ |]\notag\\
	  		 &\mathit{coat\_off \mathbin{{-}{>}} eat \mathbin{{-}{>}} request\_coat \mathbin{{-}{>}} coat\_on \mathbin{{-}{>}} SKIP}\notag
\end{align}

The system comprises six possible events (\emph{coat\_on}, \emph{coat\_off}, among others), and its behaviour is described by the parallel synchronisation of the attendant's and the customer's behaviours, requiring that they need to synchronise on the execution of the following events: \emph{coat\_off}, \emph{request\_coat}, and \emph{coat\_on}. Regarding the other events, the attendant and the customer are free to perform them as they wish.

We can declare such a system in \CSPcoq{} by defining a specification, which consists of lists of channels and processes. This specification must also abide by a set of contextual rules that will be discussed further in this work.

\begin{coqdoccode}
	\coqdocnoindent
	\coqdockw{Definition} \coqdocvar{example} : \coqdocvar{specification}.\coqdoceol
	\coqdocnoindent
	\coqdockw{Proof}.\coqdoceol
	\coqdocindent{1.00em}
	\coqdocvar{solve\_spec\_ctx\_rules} (\coqdoceol
	\coqdocindent{2.00em}
	\coqdocvar{Build\_Spec}\coqdoceol
	\coqdocindent{2.00em}
	[ \coqdocvar{Channel} \{\{``coat\_off'', ``coat\_on'', ``request\_coat'', ``retrieve'', ``store'', ``eat''\}\} ]\coqdoceol
	\coqdocindent{2.00em}
	[ ``SYSTEM'' ::=\coqdoceol
	\coqdocindent{3.00em}
	``coat\_off'' -{}-> ``store'' -{}-> ``request\_coat'' -{}-> ``retrieve'' -{}-> ``coat\_on'' -{}-> \coqdocvar{SKIP}\coqdoceol
	\coqdocindent{3.00em}
	[| \{\{``coat\_off'', ``request\_coat'', ``coat\_on''\}\} |]\coqdoceol
	\coqdocindent{3.00em}
	``coat\_off'' -{}-> ``eat'' -{}-> ``request\_coat'' -{}-> ``coat\_on'' -{}-> \coqdocvar{SKIP} ]\coqdoceol
	\coqdocindent{1.00em}
	).\coqdoceol
	\coqdocnoindent
	\coqdockw{Defined}.\coqdoceol
\end{coqdoccode}

As one can see, the main syntax of \CSPcoq{} is close to that of \CSPM{}. It is also important to emphasise that the tactic \emph{solve\_spec\_ctx\_rules} proves the aforementioned contextual rules with no user intervention (automatically). Furthermore, we can execute the following command to compute the process LTS and output the corresponding graph in the dot language:

\begin{coqdoccode}
	\coqdocnoindent
	\coqdockw{Compute} \coqdocvar{generate\_dot} (\coqdocvar{compute\_ltsR} \coqdocvar{example} ``SYSTEM'' 100).\coqdoceol
\end{coqdoccode}

\autoref{lts-example} is a visual representation of the graph outputted by this command. The image is generated using the GraphViz software. The red circle denotes the starting node, and the edges are labelled by the performed events.

\begin{figure}[htb]
	\caption[The LTS for the cloakroom example]{The LTS for the cloakroom example}
	\label{lts-example}
	\begin{center}
		\includegraphics[scale=0.5]{images/LTS.pdf}
	\end{center}
\end{figure}

% ---
\section{Main contributions}
% ---

The main contribution of this work is an initial formalisation of CSP in Coq (\CSPcoq{}), which takes into account the following aspects:

\begin{itemize}
	\item Abstract and concrete syntax for a subset of CSP operators.
	\item Contextual rules for CSP specifications.
	\item Proof automation for checking contextual rules.
	\item Operational semantics via the SOS approach.
	\item Inductive and functional definitions of labelled transition systems.
	\item Inductive and functional definitions of traces.
	\item Proof automation for checking whether a list of events is a valid trace.
	\item Formal definition of traces refinement.
	\item Traces refinement verification using QuickChick.
\end{itemize}

% ---
\section{Document structure}
% ---

Apart from this introductory chapter, in which we discuss about the motivation behind this work and its main objective, and also takes a quick look at an example that illustrates what can be done using the framework developed, this monograph contains three more chapters. The content of these chapters are detailed below:

\begin{description}
	\item [Chapter 2] Discusses fundamental concepts such as the CSP theory, the SOS approach, trace refinement and LTS representation. Additionally, this chapter introduces the Coq proof assistant and its functional language Gallina, along with an introduction to proof development (tactics) and the Ltac language, which gives support for proof automation.
	\item [Chapter 3] Provides an in-depth look at the implementation of \CSPcoq{}, including its abstract and concrete syntax, in addition to the language semantics. Furthermore, the support for visualising processes as LTSs, using the GraphViz software, is also detailed in this chapter.
	\item [Chapter 4] Concludes this monograph by presenting a comparison between the infrastructure described in this work and related solutions based on other interactive theorem provers. It also addresses possible topics for future work.
\end{description}

% ---
% Chapter 2
% ---
\chapter{Background}
\label{chapter:background}
% ---

Before jumping into the specifics of the implementation of CSP\textsubscript{Coq}, we need to understand some elements of the CSP language such as the concrete syntax and the semantics, defined in both denotational and operational models (Section~\ref{section:csp}). Beyond that, it is also important to provide an overview of what an interactive theorem prover is: the Coq proof assistant fundamentals such as tactics and the embedded Ltac language (Section~\ref{section:coq}). We also explain in this chapter and the QuickChick tool (Section~\ref{section:quickchick}), which is the Coq implementation of QuickCheck~\cite{hughes:quickcheck2000}. This chapter introduces the reader to each one of these concepts.

% ---
\section{Communicating sequential processes}
\label{section:csp}
% ---

In 1978, Tony Hoare's \emph{Communicating Sequential Processes} described a theory to help us understand concurrent systems, parallel programming and multiprocessing. More than that, it has introduced a way to decide whether a program meets its specification. This theory quickly evolved into what we know today as the CSP programming language. This language belongs to a class of notations known as process algebras, where concepts of communication and interaction are presented in an algebraic style.

% TODO Apresentar os conceitos de processo e evento (interface).

The most basic process one can define is $ \mathit{\STOP} $. Essentially, this process never interacts with the environment and its only purpose is to declare the end of an execution. In other words, it illustrates a deadlock: a state in which the process can not engage in any event or make any progress whatsoever. It could be used to describe a computer that failed booting because one of its components is damaged, or a camera that can no longer take pictures due to storage space shortage.

Another simple process is $ \mathit{\SKIP} $. It indicates that the process has reached a successful termination state, which also means that it has finished executing. We can use $ \mathit{\SKIP} $ to illustrate an athlete that has crossed the finish line, or a build for a project that has passed.

Provided these two trivial processes, $ \mathit{\STOP} $ and $ \mathit{\SKIP} $, and the knowledge of what a process interface is, we can apply a handful of CSP operators to define more descriptive processes. For example, let \emph{a} be an event in the process \emph{P} interface. One can write the new process \emph{P} as $ a \then \mathit{\STOP} $, meaning that this process behaves as $ \mathit{\STOP} $ after performing \emph{a}. This operator is known as the \emph{event prefix}, and it is pronounced as ``then''.

The choice between processes can be implemented in two different ways in CSP: externally and internally. An \emph{external choice} between two processes implies the ability to perform any event that either process can engage in. Therefore, the environment has control over the outcome of such decision. On the other hand, if the process itself is the only responsible for deciding which event from its interface will be communicated, thus which process it will resolve to, then we call it an \emph{internal choice}. Note that this is essentially the source of non-determinism. To illustrate the difference among these operators, consider the following scenario: a cafeteria can operate either by letting the costumers choose between ice cream and cake for desert, or by making this choice itself, having the clients no take on what deserts they will get. In the first configuration, the choice is external to the business, whereas it is internal in the latter.

% ---
\subsection{Structured operational semantics}
% ---

Since the early 1980s there have been three complementary approaches to understanding the semantics of CSP programs. These are algebra, where we set out laws that the syntax is assumed to satisfy, behavioral models such as traces that form the basis of refinement relations and other things, and operational models, which try to understand all the actions and decisions that process implementations can make as they proceed.

The operational semantics of CSP treats the CSP language itself as a (large!) LTS. It allows us to compute the initial events of any  process, and what processes it might become after each such event. By selecting one of these actions and repeating the procedure, we can explore the state space of the process we started with. The operational semantics gives a one-state-at-a-time recipe for computing the transition system picture of any process. It is traditional to present operational semantics as a logical inference system: Plotkin’s SOS, or Structured Operational Semantics style. A process has a given action if and only if that is deducible from the rules given.

% ---
\subsection{Traces refinement}
% ---

Imagine you are interacting with a CSP process. The most basic record you might make of what happens is to write down the trace of events that occur: the sequence of communications between you (the environment) and the process. In general, a trace might be finite or infinite: finite either because the observation was terminated or because the process and environment reach a point where they cannot agree on any event; infinite when the observation goes on for ever and infinitely many events are transacted. Traces are typical of the sort of behaviors we use to build models of CSP processes: they are clearly observable by the environment interacting with the process, and each of them is the record of single interaction with the process. We will see more details of the traces model in the next chapter, but we now introduce a fundamental way in which we can specify the correctness of a CSP process. Using traces(P) to denote P’s finite traces we can write

and read this as \emph{Q trace-refines P}. In other words, every trace of \emph{Q} is a trace of \emph{P}.

% ---
\subsection{Machine-readable version of CSP}
% ---

The main purpose of CSP is to describe communicating and interacting processes. But in order to make it useful in practice we have added quite a rich language of sub-process objects: as we saw in Chap.8, CSPM contains a functional programming language to describe and manipulate things like events and process parameters.

% ---
\section{The Coq proof assistant}
\label{section:coq}
% ---

Coq is a formal proof management system. It provides a formal language to write mathematical definitions, executable algorithms and theorems together with an environment for semi-interactive development of machine-checked proofs. Typical applications include the certification of properties of programming languages, the formalization of mathematics, and teaching. 

% ---
\subsection{Building proofs}
% ---

As a proof development system, Coq provides interactive proof methods, decision and semi-decision algorithms, and a tactic language for letting the user define its own proof methods. Proof development in Coq is done through a language of tactics that allows a user-guided proof process.

% ---
\subsection{The tactics language}
% ---

Ltac is the tactic language for Coq. It provides the user with a high-level “toolbox” for tactic creation, allowing one to build complex tactics by combining existing ones with constructs such as conditionals and looping.

% ---
\section{QuickChick}
\label{section:quickchick}
% ---

QuickChick is a set of tools and techniques for combining randomized property-based testing with formal specification and proof in the Coq ecosystem.

% ---
% Chapter 3
% ---
\chapter{A theory for CSP in Coq}
\label{chapter:csp_coq}
% ---

The \autoref{chapter:background} provided an overview of the essential concepts for understanding the implementation of the \CSPcoq{} language, setting the scene for an in-depth explanation of the language developed in this work. That being said, \autoref{chapter:csp_coq} will explain how each of these concepts combined together made it possible to develop a theory of communicating sequential processes in Coq proof assistant. The \autoref{section:syntax} discusses the implementation of the language's abstract and concrete syntax, whereas \autoref{section:sos} explains how the SOS style of defining language semantics translates into Coq as an inductive declaration. Furthermore, \autoref{section:lts} provides details on both functional and inductive definitions of LTS, along with further explanation on the GraphViz software integration. Finally, \autoref{section:traces} guides the reader through the definition of traces refinement as an executable property and also presents randomized testing based on this property using QuickChick plugin.

% ---
\section{Syntax}
\label{section:syntax}
% ---

The \CSPcoq{} language provides support to all process constructors and operations discussed in \autoref{section:csp}. Those include the processes $ \mathit{\STOP} $ and $ \mathit{\SKIP} $, and the operations event prefix, external choice, internal choice, alphabetized parallel, generalized parallel, interleave, sequential composition, and event hiding. This section introduces the reader to the implementation of both the abstract and concrete syntax of the \CSPcoq{} language in Coq.

% ---
\subsection{Abstract syntax}
% ---

In order to define all these CSP operations in Coq, we declared the following inductive types: \emph{event}, \emph{event\_tau\_tick}, \emph{channel}, \emph{alphabet}, \emph{proc\_body}, and \emph{proc\_def}. The type \emph{event} represents all external events. As we said before, external events are all events that are neither internal ($ \tau $) nor indicate termination ($ \tick $). On the other hand, the type \emph{event\_tau\_tick} provides not only a constructor for the external events, but also for the especial events $ \tau $ and $ \tick $.

For describing a set of external events, we have the \emph{channel} and \emph{alphabet} types. They are syntactically equivalent, the difference between these two constructors being purely semantic: the \emph{channel} type is used to account for all external events that may be communicated in a \CSPcoq{} specification, while the constructor provided by the \emph{alphabet} type is applied, for example, to enumerate all external events in a process interface in an alphabetized parallel combination.

The \emph{proc\_body} and \emph{proc\_def} types describe, respectively, the constructors for the operations we mentioned earlier in the beginning of this subsection, and the process attribution statement. The constructors available for the \emph{proc\_body} type allow us to combine processes in order to create more complex ones, while the \emph{proc\_def} type provides a constructor that makes it possible to identify a process by a string, and that is, to give it a name.

The last element from the CSP syntax that we described is the \emph{specification} type. A CSP specification can be perceived as a file containing multiple channels of events and process declarations. Ultimately, it is a context that holds information such as all the events that can be performed, and all process and corresponding definitions that compose a system. The way we introduce this concept in \CSPcoq{} is via a record type. Records are constructions that allow the definition of a set of attributes and propositions that must be satisfied in order to successfully define this structure.

To illustrate the abstract syntax defined via inductive types in Coq, consider the \CSPM{} processes
\
\begin{flushleft}
	PRINTER := accept -> print -> STOP

	PARKING\_PERMIT\_MCH := TICKET [ \{cash, ticket\} || \{cash, change\} ] CHANGE
\end{flushleft}
\
and their representations in \CSPcoq{} language

\begin{flushleft}
	Proc ``PRINTER'' (ProcPrefix (Event ``accept'') (ProcPrefix (Event ``print'') STOP))
\end{flushleft}

\begin{tabbing}
	\hspace*{1em}\= \hspace*{2em} \= \kill
	Proc ``PARKING\_PERMIT\_MCH'' (\\
	\>	ProcAlphaParallel (ProcRef ``TICKET'') (ProcRef ``CHANGE'')\\
	\>	(Alphabet (set\_add event\_dec ``cash'' (set\_add event\_dec ``ticket'' (empty\_set event))))\\
	\>	(Alphabet (set\_add event\_dec ``cash'' (set\_add event\_dec ``change'' (empty\_set event))))\\
	)
\end{tabbing}

As one might notice from the examples above, the abstract syntax -- though it dictates how well-formed expressions are constructed -- is not a pleasant way of writing statements or at least reading them. For that matter, we need a more convenient notation. One that resembles the \CSPM{} operators and, therefore, facilitates both implementation and understanding of \CSPcoq{} processes and specifications.

% ---
\subsection{Concrete syntax}
% ---

In order to define a more appropriate notation for the \CSPcoq{} language, the proof assistant's \coqdockw{Notation} command was used. This command allows the declaration of a new symbolic notation for an existing definition. The examples bellow demonstrate how this command is used to assign symbols (operators) to previously defined constructors.

\begin{coqdoccode}
	\coqdocnoindent
	\coqdockw{Notation} ``a \texttt{-{}-}> P'' := (\coqdocvar{ProcPrefix} \coqdocvar{a} \coqdocvar{P}) (\coqdoctac{at} \coqdockw{level} 80, \coqdoctac{right} \coqdockw{associativity}).\coqdoceol
	\coqdocnoindent
	\coqdockw{Notation} ``P [] Q'' := (\coqdocvar{ProcExtChoice} \coqdocvar{P} \coqdocvar{Q}) (\coqdoctac{at} \coqdockw{level} 90, \coqdoctac{left} \coqdockw{associativity}).\coqdoceol
	\coqdocnoindent
	\coqdockw{Notation} ``P [[ A \symbol{92}\symbol{92} B ]] Q'' := (\coqdocvar{ProcAlphaParallel} \coqdocvar{P} \coqdocvar{Q} (\coqdocvar{Alphabet} \coqdocvar{A}) (\coqdocvar{Alphabet} \coqdocvar{B})) (\coqdoctac{at} \coqdockw{level} 90, \coqdockw{no} \coqdockw{associativity}).\coqdoceol
\end{coqdoccode}

Along with the assignment of a notation symbol, we can specify its \emph{precedence level} and its \emph{associativity}. The precedence level helps Coq parse compound expressions, whereas the associativity setting helps to disambiguate expressions containing multiple occurrences of the same symbol. Coq uses precedence levels from 0 to 100, and left, right, or no associativity.

In the command lines above, the prefix operation has the higher precedence among all three operators and associates to the right. On the other hand, the external choice has a left associativity while the alphabetized parallel operator does not associates at all, meaning that brackets are necessary to create a compound expression with multiple parallel operations.

Then, we can use these symbols to rewrite the process examples from the previous subsection in a much more friendly and recognizable way:

\begin{flushleft}
	``PRINTER'' ::= ``accept'' \texttt{-{}-}> ``print'' \texttt{-{}-}> STOP

	``MACHINE'' ::= ProcRef ``TICKET'' [[ \{\{``cash'', ``ticket''\}\} \textbackslash\textbackslash \ \{\{``cash'', ``change''\}\} ]] ProcRef ``CHANGE''
\end{flushleft}

The \autoref{tab:cspm-csp_coq} displays a comparison between the \CSPM{} operators we discussed and the \CSPcoq{} language concrete syntax, achieved via Coq's \coqdockw{Notation} command:

\begin{table}[htb]
	\begin{center}
		\caption[The \CSPcoq{} concrete syntax]{The \CSPcoq{} concrete syntax.}
		\label{tab:cspm-csp_coq}
		\begin{tabular}{ |l|c|c| }
			\hline
			Constructor & \CSPM{} & \CSPcoq{} \\
			\hline
			Stop & STOP & STOP \\ [0.5ex]
			Skip & SKIP & SKIP \\ [0.5ex]
			Event prefix & e -> P & e \texttt{-{}-}> P \\  [0.5ex]
			External choice & P [] Q & P [] Q \\  [0.5ex]
			Internal choice & P |$ \sim $| Q & P |$ \sim $| Q \\ [0.5ex]
			Alphabetized parallel & P [A || B] Q & P [[A \textbackslash\textbackslash \ B]] Q \\ [0.5ex]
			Generalized parallel & P [| A |] Q & P [| A |] Q \\ [0.5ex]
			Interleave & P ||| Q & P ||| Q \\ [0.5ex]
			Sequential composition & P ; Q & P ;; Q \\ [0.5ex]
			Event hiding & P \textbackslash \ A & P \textbackslash \ A \\ [0.5ex]
			Process definition & P := Q & P ::= Q \\ [0.5ex]
			Process reference & P := e -> P & P ::= e \texttt{-{}-}> ProcRef ``P'' \\ [0.5ex]
			\hline
		\end{tabular}
	\end{center}
\end{table}

% ---
\section{Structured operational semantics}
\label{section:sos}
% ---

Now that we have a good understanding of not only how the abstract syntax but also a convenient notation of the \CSPcoq{} language were implemented in the proof assistant, it is time for us to discuss language semantics in Coq. All the declarations presented in this section are based on the inference rules discussed in \autoref{subsection:sos}. More specifically, we will address how those SOS rules from the previous chapter were ported into Coq's environment.

Recall the inductive definition of the evenness property exemplified in \autoref{section:coq}. In that example, it was possible to rewrite the inference rules for such property in terms of an inductive declaration in Coq, where each rule was translated into a propositional statement, more specifically, a logical implication. We will use the same approach to define the semantic rules of the \CSPcoq{} language.

Initially, a notation was defined to represent the SOS relation, in order to increase the readability of the inductive declaration. Thus, this new notation could be used in the constructors of the relational definition. The following Coq command line, similarly to the ones in the previous section, creates the infix notation ``S \# P // a ==> Q'', which can be pronounced ``in the specification \emph{S}, the process \emph{P}, after communicating \emph{a}, behaves like the process \emph{Q}'':

\begin{coqdoccode}
	\coqdocnoindent
	\coqdockw{Reserved Notation} ``S '\#' P '//' a '==>' Q'' (\coqdoctac{at} \coqdockw{level} 150, \coqdoctac{left} \coqdockw{associativity}).\coqdoceol
\end{coqdoccode}

To exemplify the usage of this notation inside the inductive definition, lets revisit the inference rule for the prefix operator:

\begin{prooftree}
	\AxiomC{}
	\UnaryInfC{$ (a \then P) \trans(2)[a] P $}
\end{prooftree}

This structure can be translated into Coq as a single constructor in the SOS relation:

\begin{coqdoccode}
	\coqdocnoindent
	\ensuremath{|} \coqdocvar{prefix\_rule} (\coqdocvar{S} : \coqdocvar{specification}) (\coqdocvar{P} : \coqdocvar{proc\_body}) (\coqdocvar{a} : \coqdocvar{event}) :\coqdoceol
	\coqdocindent{1.00em}
	\coqdocvar{S} \# (\coqdocvar{a} \texttt{-{}-}> \coqdocvar{P}) // \coqdocvar{Event} \coqdocvar{a} ==> \coqdocvar{P}\coqdoceol
\end{coqdoccode}

As far as the notation goes, this constructor can be interpreted as follows: given a specification $ S $, a process $ P $, and an event $ a $, in the context of $ S $, the process $ a \ \texttt{-{}-}> P $, after communicating event $ a $, behaves as the process $ P $. In other words, it is always true that, in the prefix operation $ a \then P $, after $ a $ is performed, the resulting process is $ P $.

Consider the following examples of constructors from the SOS relation. They refer to the external choice and alphabetized parallelism operations respectively. Note that each one of these operations need more than one inference rule in order to fully describe their semantics, as explained in \autoref{subsection:sos}. However, the definitions we are about to discuss only consider two of these rules: one that evaluates the external choice to the left side operand, and the joint step rule of the alphabetized parallelism operation, that represents the synchronization event between the processes.

\begin{coqdoccode}
	\coqdocnoindent
	\ensuremath{|} \coqdocvar{ext\_choice\_left\_rule} (\coqdocvar{S} : \coqdocvar{specification}) (\coqdocvar{P} \coqdocvar{Q} : \coqdocvar{proc\_body}) :\coqdoceol
	\coqdocindent{1.00em}
	\coqdockw{\ensuremath{\forall}} (\coqdocvar{P'} : \coqdocvar{proc\_body}) (\coqdocvar{a} : \coqdocvar{event\_tau\_tick}),\coqdoceol
	\coqdocindent{3.00em}
	\ensuremath{\lnot} \coqdocvar{eq} \coqdocvar{a} \coqdocvar{Tau} \ensuremath{\rightarrow}\coqdoceol
	\coqdocindent{3.00em}
	(\coqdocvar{S} \# \coqdocvar{P} // \coqdocvar{a} ==> \coqdocvar{P'}) \ensuremath{\rightarrow}\coqdoceol
	\coqdocindent{3.00em}
	(\coqdocvar{S} \# \coqdocvar{P} [] \coqdocvar{Q} // \coqdocvar{a} ==> \coqdocvar{P'})\coqdoceol
	\coqdocnoindent
	\ensuremath{|} \coqdocvar{alpha\_parall\_joint\_rule} (\coqdocvar{S} : \coqdocvar{specification}) (\coqdocvar{P} \coqdocvar{Q} : \coqdocvar{proc\_body}) (\coqdocvar{A} \coqdocvar{B} : \coqdoctac{set} \coqdocvar{event}) :\coqdoceol
	\coqdocindent{1.00em}
	\coqdockw{\ensuremath{\forall}} (\coqdocvar{P'} \coqdocvar{Q'} : \coqdocvar{proc\_body}) (\coqdocvar{a} : \coqdocvar{event}),\coqdoceol
	\coqdocindent{3.00em}
	\coqdocvar{set\_In} \coqdocvar{a} (\coqdocvar{set\_inter} \coqdocvar{event\_dec} \coqdocvar{A} \coqdocvar{B}) \ensuremath{\rightarrow}\coqdoceol
	\coqdocindent{3.00em}
	(\coqdocvar{S} \# \coqdocvar{P} // \coqdocvar{Event} \coqdocvar{a} ==> \coqdocvar{P'}) \ensuremath{\rightarrow}\coqdoceol
	\coqdocindent{3.00em}
	(\coqdocvar{S} \# \coqdocvar{Q} // \coqdocvar{Event} \coqdocvar{a} ==> \coqdocvar{Q'}) \ensuremath{\rightarrow}\coqdoceol
	\coqdocindent{3.00em}
	\coqdocvar{S} \# \coqdocvar{P} [[ \coqdocvar{A} \symbol{92}\symbol{92} \coqdocvar{B} ]] \coqdocvar{Q} // \coqdocvar{Event} \coqdocvar{a} ==> \coqdocvar{P'} [[ \coqdocvar{A} \symbol{92}\symbol{92} \coqdocvar{B} ]] \coqdocvar{Q'}\coqdoceol
\end{coqdoccode}

The \coqdocvar{ext\_choice\_left\_rule} constructor encodes the inference rule that solves the external choice operation for the left operand. As we saw earlier, this behavior is described in the SOS by the rule \
\
\begin{prooftree}
	\AxiomC{$ P \trans(2)[a] P' $}
	\RightLabel{\quad ($ a \neq \tau $)}
	\UnaryInfC{$ P \extchoice Q \trans(2)[a] P' $}
\end{prooftree}
\
which translates into the logical proposition

\begin{coqdoccode}
	\coqdocnoindent
	\ensuremath{\lnot} \coqdocvar{eq} \coqdocvar{a} \coqdocvar{Tau} \ensuremath{\rightarrow}
	(\coqdocvar{S} \# \coqdocvar{P} // \coqdocvar{a} ==> \coqdocvar{P'}) \ensuremath{\rightarrow}
	(\coqdocvar{S} \# \coqdocvar{P} [] \coqdocvar{Q} // \coqdocvar{a} ==> \coqdocvar{P'})\coqdoceol
\end{coqdoccode}

In this statement, we can see that the first term corresponds to the side condition of the inference rule, which guarantees that the event in question is not the internal event $ \tau $. The second term is the main premise of the rule, ensuring that it is possible for the event $ a $ to evolve the process $ P $ into $ P' $ in the specification $ S $. Together, the side condition and hypothesis establish the necessary conditions to resolve the external choice operation to the left hand operand.

Similarly, the inference rule that defines the synchronous communication of an event by two processes combined by the alphabetized parallelism operation, is described in Coq by the constructor \coqdocvar{alpha\_parall\_joint\_rule}. This rule, as we can see from its sequent notation below, has a side condition which guarantees that the event in question belongs to the intersection of the processes interfaces. Furthermore, it has two premises, which ensure this event is able to evolve both operands of the combination, that is, the processE  on the left and right side of the parallelism operator can communicate the event.

\begin{prooftree}
	\AxiomC{$ P \trans(2)[a] P' $}
	\AxiomC{$ Q \trans(2)[a] Q' $}
	\RightLabel{\quad ($ a \in A^{\tick} \cap B^{\tick} $)}
	\BinaryInfC{$ P \parallel[A][B] Q \trans(2)[a] P' \parallel[A][B] Q' $}
\end{prooftree}

The side condition and the premises are rewritten in the inductive definition as antecedents of a logical implication:

\begin{coqdoccode}
	\coqdocnoindent
	\coqdocvar{set\_In} \coqdocvar{a} (\coqdocvar{set\_inter} \coqdocvar{event\_dec} \coqdocvar{A} \coqdocvar{B}) \ensuremath{\rightarrow}\coqdoceol
	\coqdocindent{1.00em}
	(\coqdocvar{S} \# \coqdocvar{P} // \coqdocvar{Event} \coqdocvar{a} ==> \coqdocvar{P'}) \ensuremath{\rightarrow}\coqdoceol
	\coqdocindent{1.00em}
	(\coqdocvar{S} \# \coqdocvar{Q} // \coqdocvar{Event} \coqdocvar{a} ==> \coqdocvar{Q'}) \ensuremath{\rightarrow}\coqdoceol
	\coqdocindent{1.00em}
	\coqdocvar{S} \# \coqdocvar{P} [[ \coqdocvar{A} \symbol{92}\symbol{92} \coqdocvar{B} ]] \coqdocvar{Q} // \coqdocvar{Event} \coqdocvar{a} ==> \coqdocvar{P'} [[ \coqdocvar{A} \symbol{92}\symbol{92} \coqdocvar{B} ]] \coqdocvar{Q'}\coqdoceol
\end{coqdoccode}

Note that this constructor does not consider the $ \tick $ event as a possible synchronization event between the processes, even though it is included in the inference rule (side condition) we saw above. The reason for this is that we defined another constructor, \coqdocvar{alpha\_parall\_tick\_joint\_rule}, dedicated exclusively to this joint step allowed by the termination event.

The last example of constructor we want to highlight from the inductive definition of the SOS is the \emph{process reference} operation. This constructor defines the rules for unfolding a process definition inside the body of another process:

\begin{coqdoccode}
	\coqdocnoindent
	\ensuremath{|} \coqdocvar{reference\_rule} (\coqdocvar{S} : \coqdocvar{specification}) (\coqdocvar{P} : \coqdocvar{proc\_body}) (\coqdocvar{name} : \coqdocvar{string}) :\coqdoceol
	\coqdocindent{1.00em}
	\coqdockw{\ensuremath{\forall}} (\coqdocvar{Q} : \coqdocvar{proc\_body}),\coqdoceol
	\coqdocindent{3.00em}
	\coqdocvar{eq} \coqdocvar{P} (\coqdocvar{ProcRef} \coqdocvar{name}) \ensuremath{\rightarrow}\coqdoceol
	\coqdocindent{3.00em}
	\coqdocvar{eq} (\coqdocvar{get\_proc\_body} \coqdocvar{S} \coqdocvar{name}) (\coqdocvar{Some} \coqdocvar{Q}) \ensuremath{\rightarrow}\coqdoceol
	\coqdocindent{3.00em}
	\coqdocvar{S} \# \coqdocvar{P} // \coqdocvar{Tau} ==> \coqdocvar{Q}\coqdoceol
\end{coqdoccode}

As we can see, there are two premises for successfully unfolding a definition. First, the process that needs unfolding must be of the form \coqdocvar{ProcRef} \coqdocvar{name}, where \coqdocvar{ProcRef} is a constructor of the type \coqdocvar{proc\_body}, and \coqdocvar{name} is a string that identifies the process in a specification. Second, the process referenced by \coqdocvar{ProcRef} \coqdocvar{name} must be defined in the specification. To that end, the function \coqdocvar{get\_proc\_body} searches for this definition inside the specification, while the equality ensures this search results in ``$ Some \ Q $'', where $ Q $ is a process body. It is important to emphasize that such definition differs from the one implemented by the FDR tool. This approach will allow an ill-formed recursion such as $ P := P $ to be declared in \CSPcoq{}, introducing a $ \tau $ action to represent the effort of unfolding the definition and, therefore, increasing the process LTS size.

% ---
\section{Labelled transition systems}
\label{section:lts}
% ---

Now that we can define syntactically and semantically correct CSP processes in Coq through the definitions we declared so far, we will introduce yet another way of implementing processes, based on the idea of labelled transition systems. This section will cover how the LTS concept was embedded in the proof assistant via functional and inductive definitions, and how \CSPcoq{} provides support to the GraphViz software, which enables a graph visualization of CSP processes.

A labelled transition system consists of a non-empty set of states $ S $, with a designated initial state $ P_{0} $, a set of labels $ L $, and a ternary relation $ P \trans[x] Q $ meaning that the state $ P $ can perform an action labelled $ x $ and move to state $ Q $. In Coq, we defined a LTS in two different and complementary ways: a computable function and a relation.

To illustrate these definitions, consider the following \CSPcoq{} specification:

\begin{coqdoccode}
	\coqdocnoindent
	\coqdockw{Definition} \coqdocvar{SPEC} : \coqdocvar{specification}.\coqdoceol
	\coqdocnoindent
	\coqdockw{Proof}.\coqdoceol
	\coqdocindent{1.00em}
	\coqdocvar{solve\_spec\_ctx\_rules} (\coqdoceol
	\coqdocindent{2.00em}
	\coqdocvar{Build\_Spec}\coqdoceol
	\coqdocindent{2.00em}
	[ \coqdocvar{Channel} \{\{``a'', ``b'', ``c''\}\} ]\coqdoceol
	\coqdocindent{2.00em}
	[ ``P'' ::= (``a'' -{}-> ``b'' -{}-> \coqdocvar{STOP}) [] (``c'' -{}-> \coqdocvar{STOP}) ]\coqdoceol
	\coqdocindent{1.00em}
	).\coqdoceol
	\coqdocnoindent
	\coqdockw{Defined}.\coqdoceol
\end{coqdoccode}

Initially, the process ``P'' can perform either ``a'' or ``c''. If ``a'' gets communicated, the external choice resolves to the process ``b'' -{}-> STOP, which can then perform ``b'' and finish the execution. On the other hand, if the event ``c'' gets communicated instead, then the combination resolves to the process STOP and also terminates. Therefore, we can list three transitions for the process ``P'': from state (``a'' -{}-> ``b'' -{}-> STOP) [] (``c'' -{}-> STOP) to state ``b'' -{}-> STOP with event ``a'', from state (``a'' -{}-> ``b'' -{}-> STOP) [] (``c'' -{}-> STOP) to state STOP with event ``c'', and from state ``b'' -{}-> STOP to state STOP with event ``b''. To better describe these transitions, we will be using the 3-tuple notation ($ P $, $ a $, $ Q $), where $ P $ is the source state, $ a $ is the action (event), and $ Q $ is the target state.

We can apply the same intuition from this example in the implementation of a functional definition of LTS in Coq: starting with the initial state, compute all the immediate transitions available from this state, meaning all 3-tuples where the target states can be reached in one step from $ S_{0} $. Mark the initial state as ``visited'' and any other state discovered in the previous step as ``not visited''. Then, repeat this step for each unvisited state until there are no states to be visited anymore, keeping track of the states you have already visited and the ones that haven't been visited yet.

Let \coqdocvar{compute\_ltsR} be the function yielded by this algorithm, the following Coq command outputs the set of transitions from the LTS of the process ``P'':

\begin{coqdoccode}
	\coqdocnoindent
	\coqdockw{Compute} \coqdocvar{compute\_ltsR} \coqdocvar{SPEC} ``P'' 1000.\coqdoceol
\end{coqdoccode}

\begin{flushleft}
	Output:
\end{flushleft}

\begin{tabbing}
	\hspace*{2.5em}\= \hspace*{2em} \= \kill
	= Some\\
	\>	[(``a'' -{}-> ``b'' -{}-> STOP [] ``c'' -{}-> STOP, Event ``a'', ``b'' -{}-> STOP);\\
	\>	(``a'' -{}-> ``b'' -{}-> STOP [] ``c'' -{}-> STOP, Event ``c'', STOP);\\
	\>	(``b'' -{}-> STOP, Event ``b'', STOP)]\\
	: option (set transition)
\end{tabbing}

We can check whether this set of transitions is indeed the LTS of the process in question. For that \textellipsis \ This is actually the same idea behind the inductive definition of the LTS in \CSPcoq{}:

\begin{coqdoccode}
	\coqdocnoindent
	\coqdockw{Inductive} \coqdocvar{ltsR'} :\coqdoceol
	\coqdocindent{1.00em}
	\coqdocvar{specification} \ensuremath{\rightarrow} \coqdoceol
	\coqdocindent{1.00em}
	\coqdoctac{set} \coqdocvar{transition} \ensuremath{\rightarrow} \coqdoceol
	\coqdocindent{1.00em}
	\coqdoctac{set} \coqdocvar{proc\_body} \ensuremath{\rightarrow} \coqdoceol
	\coqdocindent{1.00em}
	\coqdoctac{set} \coqdocvar{proc\_body} \ensuremath{\rightarrow} \coqdoceol
	\coqdocindent{1.00em}
	\coqdockw{Prop} :=\coqdoceol
	\coqdocindent{1.00em}
	\ensuremath{|} \coqdocvar{lts\_empty\_rule} (\coqdocvar{S} : \coqdocvar{specification}) (\coqdocvar{visited} : \coqdoctac{set} \coqdocvar{proc\_body}) :\coqdoceol
	\coqdocindent{3.00em}
	\coqdocvar{ltsR'} \coqdocvar{S} \coqdocvar{nil} \coqdocvar{nil} \coqdocvar{visited}\coqdoceol
	\coqdocindent{1.00em}
	\ensuremath{|} \coqdocvar{lts\_inductive\_rule}\coqdoceol
	\coqdocindent{4.00em}
	(\coqdocvar{S} : \coqdocvar{specification})\coqdoceol
	\coqdocindent{4.00em}
	(\coqdocvar{T} : \coqdoctac{set} \coqdocvar{transition})\coqdoceol
	\coqdocindent{4.00em}
	(\coqdocvar{P} : \coqdocvar{proc\_body})\coqdoceol
	\coqdocindent{4.00em}
	(\coqdocvar{tl} \coqdocvar{visited} : \coqdoctac{set} \coqdocvar{proc\_body}) :\coqdoceol
	\coqdocindent{3.00em}
	\coqdockw{let} \coqdocvar{T'} := \coqdocvar{transitions\_from} \coqdocvar{P} \coqdocvar{T} \coqdoctac{in}\coqdoceol
	\coqdocindent{3.00em}
	\coqdockw{let} \coqdocvar{T'{}'} := \coqdocvar{set\_diff} \coqdocvar{transition\_eq\_dec} \coqdocvar{T} \coqdocvar{T'} \coqdoctac{in}\coqdoceol
	\coqdocindent{3.00em}
	\coqdockw{let} \coqdocvar{visited'} := \coqdocvar{set\_add} \coqdocvar{proc\_body\_eq\_dec} \coqdocvar{P} \coqdocvar{visited} \coqdoctac{in}\coqdoceol
	\coqdocindent{3.00em}
	\coqdockw{let} \coqdocvar{to\_visit} := \coqdocvar{set\_diff} \coqdocvar{proc\_body\_eq\_dec}\coqdoceol
	\coqdocindent{12.00em}
	(\coqdocvar{set\_union} \coqdocvar{proc\_body\_eq\_dec} \coqdocvar{tl} (\coqdocvar{target\_proc\_bodies} \coqdocvar{T'}))\coqdoceol
	\coqdocindent{12.00em}
	\coqdocvar{visited'} \coqdoctac{in}\coqdoceol
	\coqdocindent{3.00em}
	(\coqdockw{\ensuremath{\forall}} (\coqdocvar{a} : \coqdocvar{event\_tau\_tick}) (\coqdocvar{P'} : \coqdocvar{proc\_body}),\coqdoceol
	\coqdocindent{4.50em}
	(\coqdocvar{S} \# \coqdocvar{P} // \coqdocvar{a} ==> \coqdocvar{P'}) \ensuremath{\leftrightarrow} \coqdocvar{In} (\coqdocvar{P},\coqdocvar{a},\coqdocvar{P'}) \coqdocvar{T'}) \ensuremath{\rightarrow}\coqdoceol
	\coqdocindent{3.00em}
	\coqdocvar{ltsR'} \coqdocvar{S} \coqdocvar{T'{}'} \coqdocvar{to\_visit} \coqdocvar{visited'} \ensuremath{\rightarrow}\coqdoceol
	\coqdocindent{3.00em}
	\coqdocvar{ltsR'} \coqdocvar{S} \coqdocvar{T} (\coqdocvar{P} :: \coqdocvar{tl}) \coqdocvar{visited}.\coqdoceol
	\coqdocemptyline
	\coqdocnoindent
	\coqdockw{Definition} \coqdocvar{ltsR} (\coqdocvar{S} : \coqdocvar{specification}) (\coqdocvar{name} : \coqdocvar{string}) (\coqdocvar{T} : \coqdoctac{set} \coqdocvar{transition}) : \coqdockw{Prop} :=\coqdoceol
	\coqdocindent{1.00em}
	\coqdockw{match} \coqdocvar{get\_proc\_body} \coqdocvar{S} \coqdocvar{name} \coqdockw{with}\coqdoceol
	\coqdocindent{1.00em}
	\ensuremath{|} \coqdocvar{Some} \coqdocvar{body} \ensuremath{\Rightarrow} \coqdocvar{NoDup} \coqdocvar{T} \ensuremath{\land} \coqdocvar{ltsR'} \coqdocvar{S} \coqdocvar{T} [\coqdocvar{body}] \coqdocvar{nil}\coqdoceol
	\coqdocindent{1.00em}
	\ensuremath{|} \coqdocvar{None} \ensuremath{\Rightarrow} \coqdocvar{False}\coqdoceol
	\coqdocindent{1.00em}
	\coqdockw{end}.\coqdoceol
\end{coqdoccode}

% ---
\subsection{GraphViz integration}
% ---

GraphViz is an open source graph visualization software that takes descriptions of graphs in a simple text language -- in our case, the DOT language -- and make diagrams in useful formats, such as images, SVG, and PDF. Moreover, this tool allows the customization of these diagrams, such as options for colors, fonts, hyperlinks, and custom node shapes. Providing support for this software enables \CSPcoq{} programmers to generate a graph from the description of a LTS with one line of command. That way, the behavior of a process may be inspected and analyzed more easily, since a visual representation will be available.

In order to build the description of a graph in DOT language, it is necessary to define a mapping of the constructors of \coqdocvar{event\_tau\_tick} and \coqdocvar{proc\_body} types to the \coqdocvar{string} type. That way, it will be possible to convert each tuple from the LTS into a statement that corresponds to a transition of a graph in DOT syntax. One way to achieve this in Coq is through the \coqdockw{Coercion} command, which assigns a conversion function that maps an expression of one type into another.

Consider the following coercion from type \coqdocvar{event\_tau\_tick} to type \coqdocvar{string}:

\begin{coqdoccode}
	\coqdocnoindent
	\coqdockw{Definition} \coqdocvar{event\_tau\_tick\_to\_str} (\coqdocvar{e} : \coqdocvar{event\_tau\_tick}) : \coqdocvar{string} :=\coqdoceol
	\coqdocindent{1.00em}
	\coqdockw{match} \coqdocvar{e} \coqdockw{with}\coqdoceol
	\coqdocindent{1.00em}
	\ensuremath{|} \coqdocvar{Tau} \ensuremath{\Rightarrow} ``$ \tau $''\coqdoceol
	\coqdocindent{1.00em}
	\ensuremath{|} \coqdocvar{Tick} \ensuremath{\Rightarrow} ``$ \tick $''\coqdoceol
	\coqdocindent{1.00em}
	\ensuremath{|} \coqdocvar{Event} \coqdocvar{a} \ensuremath{\Rightarrow} \coqdocvar{a}\coqdoceol
	\coqdocindent{1.00em}
	\coqdockw{end}.\coqdoceol
	\coqdocemptyline
	\coqdocnoindent
	\coqdockw{Coercion} \coqdocvar{event\_tau\_tick\_to\_str} : \coqdocvar{event\_tau\_tick} >-> \coqdocvar{string}.\coqdoceol
\end{coqdoccode}

As we can see from the code snippet above, initially, we defined a function that takes an element of type \coqdocvar{event\_tau\_tick} and returns a string representation of that element. Then, a coercion between the two types is established: from this point on, Coq will convert an element of type \coqdocvar{event\_tau\_tick} into a string whenever it is necessary.

Once the coercions are configured, we can define a function to create the description of a graph in the DOT language from the set of tuples that make up the LTS. The intuition behind this function is as follows: the first tuple to be processed has its starting state identified as the graph's initial node (bold and red border), and each tuple -- the first included -- is rewritten as a DOT syntax transition statement. In other words, the tuple (P, e, Q) generates the string ``<P> -> <Q> [label = <e>];'', as shown in the definitions below:

\begin{coqdoccode}
	\coqdocnoindent
	\coqdockw{Fixpoint} \coqdocvar{generate\_dot'} (\coqdocvar{lts} : \coqdoctac{set} \coqdocvar{transition}) : \coqdocvar{string} :=\coqdoceol
	\coqdocindent{1.00em}
	\coqdockw{match} \coqdocvar{lts} \coqdockw{with}\coqdoceol
	\coqdocindent{1.00em}
	\ensuremath{|} \coqdocvar{nil} \ensuremath{\Rightarrow} ``''\coqdoceol
	\coqdocindent{1.00em}
	\ensuremath{|} (\coqdocvar{P}, \coqdocvar{e}, \coqdocvar{Q}) :: \coqdocvar{tl} \ensuremath{\Rightarrow} `` <'' ++ \coqdocvar{P} ++ ``> -> <'' ++ \coqdocvar{Q} ++ ``>'' ++ `` [label=<'' ++ \coqdocvar{e} ++ ``>];'' ++ (\coqdocvar{generate\_dot'} \coqdocvar{tl})\coqdoceol
	\coqdocindent{1.00em}
	\coqdockw{end}.\coqdoceol
	\coqdocemptyline
	\coqdocnoindent
	\coqdockw{Definition} \coqdocvar{style\_initial\_state} (\coqdocvar{P} : \coqdocvar{proc\_body}) : \coqdocvar{string} := ``<'' ++ \coqdocvar{P} ++ ``> [style=bold, color=red];''.\coqdoceol
	\coqdocemptyline
	\coqdocnoindent
	\coqdockw{Definition} \coqdocvar{generate\_dot} (\coqdocvar{lts} : \coqdocvar{option} (\coqdoctac{set} \coqdocvar{transition})) : \coqdocvar{string} :=\coqdoceol
	\coqdocindent{1.00em}
	\coqdockw{match} \coqdocvar{lts} \coqdockw{with}\coqdoceol
	\coqdocindent{1.00em}
	\ensuremath{|} \coqdocvar{Some} ((\coqdocvar{P}, \coqdocvar{e}, \coqdocvar{Q}) :: \coqdocvar{tl}) \ensuremath{\Rightarrow}\coqdoceol
	\coqdocindent{2.00em}
	``digraph LTS \{ '' ++ (\coqdocvar{style\_initial\_state} \coqdocvar{P}) ++ (\coqdocvar{generate\_dot'} ((\coqdocvar{P}, \coqdocvar{e}, \coqdocvar{Q}) :: \coqdocvar{tl})) ++ `` \}''\coqdoceol
	\coqdocindent{1.00em}
	\ensuremath{|} \coqdocvar{\_} \ensuremath{\Rightarrow} ``''\coqdoceol
	\coqdocindent{1.00em}
	\coqdockw{end}.\coqdoceol
\end{coqdoccode}

To demonstrate what these definitions can achieve, recall the process MACHINE from the section 3.1.2. Here is a complete definition of that process in \CSPcoq{}:

\begin{coqdoccode}
	\coqdocnoindent
	\coqdockw{Definition} \coqdocvar{PARKING\_PERMIT\_MCH} : \coqdocvar{specification}.\coqdoceol
	\coqdocnoindent
	\coqdockw{Proof}.\coqdoceol
	\coqdocindent{1.00em}
	\coqdocvar{solve\_spec\_ctx\_rules} (\coqdoceol
	\coqdocindent{2.00em}
	\coqdocvar{Build\_Spec}\coqdoceol
	\coqdocindent{2.00em}
	[ \coqdocvar{Channel} \{\{``cash'', ``ticket'', ``change''\}\} ]\coqdoceol
	\coqdocindent{2.00em}
	[ ``TICKET'' ::= ``cash'' -{}-> ``ticket'' -{}-> \coqdocvar{ProcRef} ``TICKET''\coqdoceol
	\coqdocindent{2.00em}
	; ``CHANGE'' ::= ``cash'' -{}-> ``change'' -{}-> \coqdocvar{ProcRef} ``CHANGE''\coqdoceol
	\coqdocindent{2.00em}
	; ``MACHINE'' ::= \coqdocvar{ProcRef} ``TICKET'' [[ \{\{``cash'', ``ticket''\}\} \symbol{92}\symbol{92} \{\{``cash'', ``change''\}\} ]] \coqdocvar{ProcRef} ``CHANGE'' ]\coqdoceol
	\coqdocindent{1.00em}
	).\coqdoceol
	\coqdocnoindent
	\coqdockw{Defined}.\coqdoceol
\end{coqdoccode}

To generate the description of the graph in DOT syntax for the process MACHINE, simply run this command in Coq:

\begin{coqdoccode}
	\coqdocnoindent
	\coqdockw{Compute} \coqdocvar{generate\_dot} (\coqdocvar{compute\_ltsR} \coqdocvar{PARKING\_PERMIT\_MCH} ``MACHINE'' 1000).\coqdoceol
\end{coqdoccode}

We can then save the string output to a file named LTS.gv and run the following command in a terminal with the GraphViz software installed:

\emph{\$ dot -Nlabel=``'' -Nshape=circle -Tjpeg LTS.gv -o LTS.jpeg}

The \autoref{image:machine_lts} shows the expected output: an image in JPEG format containing a graph -- with nodes shaped as circles and no labels -- that represents the process LTS:

\begin{figure}[htb]
	\caption[Example LTS: The process MACHINE]{Example LTS: The process MACHINE.}
	\label{image:machine_lts}
	\begin{center}
		\includegraphics[scale=0.75]{images/parking_permit_mch_lts.pdf}
	\end{center}
\end{figure}

% ---
\section{Traces refinement}
\label{section:traces}
% ---

% ---
\subsection{QuickChick integration}
% ---

% ---
% Chapter 4
% ---
\chapter{Conclusions}
% ---

In this work, we embedded a subset of the CSP language in the Coq proof assistant, giving rise to the language entitled \CSPcoq{}. The abstract syntax was described through inductive types, while the concrete language relies on the concept of notations. In addition, the inductive declaration that defines operational semantics in the SOS style was also presented.

The concept of LTS was represented both in an inductive and functional approach, supporting a third-party tool that allows a custom graphic visualization of this structure. Finally, the notion of the trace of a process was declared, along with a tactic macro that automates the proof of this relation. These accomplishments led to the definition of the refinement relation according to the traces model, in addition to the implementation of two generators and one checker for this property, in order to test it using a property-based random testing plugin.

% ---
\section{Related work}
% ---

% ---
\section{Future work}
% ---

The topics listed below describe relevant activities that expand the work described in this monograph:
\begin{itemize}
	\item Extend the language \CSPcoq{} to include the remaining CSP operators. The theory of CSP, as well as the ASCII version of the language, contemplates more advanced concepts such as parameterized processes and operations like event renaming, interruption, exception, among others. It is necessary to include these concepts in order to leverage the framework developed.
	\item Prove that for every process $ P $, there is a list of transitions $ L $ such that if \emph{ltsR S P L}, then there is natural $ n $ such that \emph{compute\_ltsR S P n = Some L}. This proof guarantees the completeness of the definition \emph{compute\_ltsR}, and that is, every set of transitions that characterize the LTS of a process is computable from this definition.
	\item Define the tactic macro that automatically proves if the \emph{ltsR} relation holds for a given process. The proof for the ltsR relation can be long, and they usually follow a well-defined pattern. Creating a tactic macro would simplify tasks like checking if a set of transitions is indeed the LTS of a process.
	\item Formalize the concept of a extended trace, say \emph{extended\_traceR}, including the events $ \tau $ and $ \tick $; relate \emph{traceR} and \emph{extended\_traceR}; create tactic macro that automates the proofs for the new \emph{extended\_traceR} relation. A trace, by definition, does not include the events $ \tau $ and $ \tick $, which may sometimes simplify the actual behavior of a process. Therefore, an extended definition of trace would expose these suppressed communications.
	\item Implement verification of forbidden use of recursion in the context of hiding and parallelism operations. These operations, when appearing in a process recursion, introduce the issue of accumulated operators. For instance, the process definition $ P = (a \rightarrow P) \ \textbackslash \ \{a\} $, although syntactically correct, generates a new state in the LTS of the process $ P $ every time $ P $ is unfolded, due to the addition of one more hiding operation to the process body: $ (a \rightarrow P) \ \textbackslash \ \{a\} \trans[a] ((a \rightarrow P) \ \textbackslash \ \{a\}) \ \textbackslash \ \{a\} \trans[a] (((a \rightarrow P) \ \textbackslash \ \{a\}) \ \textbackslash \ \{a\}) \ \textbackslash \ \{a\} \trans[a] \ \dots$
	\item Define traces refinement in terms of a bi-simulation. The notion of strong bi-simulation is, in order to be equivalent, two processes must have the same set of events available immediately, with these events leading to processes that are themselves equivalent. Since it is virtually impossible to implement a function that enumerates all traces of a process -- so that we could use it to make assertions about a refinement statement -- this equivalence over LTS would be the proper way to achieve this goal.
	\item Compress the generated LTS, removing intermediate $ \tau $'s. The process unwind and sequential composition operations introduce the communication of the internal event $ \tau $ as a way to take into account the ``effort'' of unfolding a process body inside another. Omitting these communications may, not only reduce the LTS size, but also make it more simple to spot equivalent process by looking at their LTS's.
	\item Prove that for all transition lists $ L1 $ and $ L2 $, if \emph{ltsR S P L1} and \emph{ltsR S P L2}, then $ L1 $ is a permutation of $ L2 $. Since our definition of LTS is based on a list of transitions instead of a set, this proof would guarantee that there aren't two lists of transitions with different elements that consists of the LTS of the same process. In other words, this would prove the uniqueness of the transition set of a LTS.
\end{itemize}

% ----------------------------------------------------------
% ELEMENTOS PÓS-TEXTUAIS
% ----------------------------------------------------------
\postextual
% ----------------------------------------------------------

% ----------------------------------------------------------
% Referências bibliográficas
% ----------------------------------------------------------
\bibliography{references}

\end{document}