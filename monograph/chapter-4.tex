% ---
% Chapter 4
% ---
\chapter{Conclusions}
% ---

In this work, we embedded a subset of the CSP language in the Coq proof assistant, giving rise to the language entitled \CSPcoq{}. The abstract syntax was described using inductive types, while the concrete syntax relies on the concept of notations. In addition, the SOS was declared using inductively defined propositions. The concept of LTSs was represented both in an inductive and functional approach, supporting a third-party tool that allows a custom graphic visualisation of this structure.

Finally, the notion of traces of a process was declared, along with a tactic macro that automates the proof of the is-a-trace relation. These accomplishments led to the definition of the refinement relation according to the traces model, in addition to the implementation of two generators and one checker for this property, in order to test it using a property-based random testing plugin (QuickChick).

% ---
\section{Related work}
% ---

Other studies have shown how to embed the theories of many CSP models in theorem proving tools such as Isabelle \cite{nipkow:isabelle}, and then prove both general laws of CSP and other coherency properties. Particularly, we want to discuss two implementations: the CSP-Prover \cite{Roggenbach:CSP-Prover} and the Isabelle/UTP verification toolbox \cite{Woodcock:Isabelle/UTP}.

% CSP-Prover
CSP-Prover is an interactive theorem prover dedicated to refinement proofs involving CSP processes, which is based on Isabelle. It focuses on the stable failures model $ \fmodel $ as the underlying denotational semantics of CSP, including the CSP traces model $ \tmodel $ as a by-product. Consequently, CSP-Prover contains the definitions of CSP's syntax and semantics, and semi-automatic proof tactics for the verification of refinement relations.

% Isabelle/UTP
Isabelle/UTP is an implementation of Hoare and He's \emph{Unifying Theories of Programming} \cite{hoare:UTP} framework based on Isabelle/HOL. UTP is a framework for construction of denotational semantic meta-models for a variety of languages based on an alphabetised relational calculus. Therefore, this implementation in Isabelle can be used to formalise semantic building blocks for different language paradigms and languages (e.g., CSP), prove algebraic laws of programming, and then use these laws to construct program verification tools. 

The main distinguishing features of \CSPcoq{} are the support for producing graphical representations of LTSs, in addition to checking traces refinement relations using property-based testing, which may scale better than the traditional model-checking approach. However, it is important to remember that this testing-based approach is sound, but not complete.

% ---
\section{Future work}
% ---

This work is a first representation of CSP in Coq and, thus, the topics listed below describe relevant activities that extend these first results and, thus, should be addressed in future developments.

\begin{itemize}
	\item Extend \CSPcoq{} to include the remaining CSP operators (e.g., indexed forms of operators, compound events, interrupt operator, among others), along with the functional language supported by CSP$_{M}$.
	
	\item Check for invalid recursions in the presence of hiding and parallelism operations. In order to prevent the creation of LTSs with an infinite number of states, some forms of recursion is not supported when using hiding and parallelism operations.
	
	\item Provide optimisation means for compressing the LTS computed by \coqdocvar{compute\_ltsR}. For instance, the process referencing operation introduces the communication of an internal event $ \tau $ as a way to take into account the ``effort'' of unfolding a process body inside another. Omitting these communications may reduce the LTS size.
	
	\item Define a tactic similar to \coqdocvar{solve\_trace} to automate proofs involving the relation \coqdocvar{ltsR}. This tactic would automate the proof of whether a computed LTS is indeed a valid LTS for a given process.
	
	\item Consider the inductively defined proposition \coqdocvar{ltsR} as the specification of the computable definition \coqdocvar{compute\_ltsR} in order to prove the correctness of the latter.
	
	\item Consider the inductively defined proposition \coqdocvar{traceR} as the specification of the traces generator \coqdocvar{gen\_valid\_trace} in order to prove the correctness of the latter.
	
	\item Define traces refinement in terms of bi-simulation. According to the notion of strong bi-simulation, in order to be equivalent, two processes must have the same set of events available immediately, with these events leading to processes that are themselves equivalent. This would enable proving trace refinement for processes with an infinite number of traces (but a finite number of states).
	
	\item Perform empirical analyses comparing \CSPcoq{} with other tools for specifying and analysing CSP processes, such as FDR.

	\item Perform a more in-depth comparison of related work.
	
\end{itemize}