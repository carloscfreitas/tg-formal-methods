% ---
% Chapter 2
% ---
\chapter{Background}
% ---

Before jumping into the specifics of the implementation of CSP\textsubscript{Coq}, we need to understand some elements of the CSP language such as the concrete syntax and the semantics, defined in both denotational and operational models. Beyond that, it is also important to provide an overview of what an interactive theorem prover is: the Coq proof assistant fundamentals such as tactics, the embedded Ltac language and the QuickChick plugin. This chapter introduces the reader to each one of these concepts.

% ---
\section{Communicating sequential processes}
% ---

In 1978, Tony Hoare's \emph{Communicating Sequential Processes} described a theory to help us understand concurrent systems, parallel programming and multiprocessing. More than that, it has introduced a way to decide whether a program meets its specification. This theory quickly evolved into what we know today as the CSP programming language. This language belongs to a class of notations known as process algebras, where concepts of communication and interaction are presented in an algebraic style.

% TODO Apresentar os conceitos de processo e evento (interface).

The most basic process one can define is $ \mathit{\STOP} $. Essentially, this process never interacts with the environment and its only purpose is to declare the end of an execution. In other words, it illustrates a deadlock: a state in which the process can not engage in any event or make any progress whatsoever. It could be used to describe a computer that failed booting because one of its components is damaged, or a camera that can no longer take pictures due to storage space shortage.

Another simple process is $ \mathit{\SKIP} $. It indicates that the process has reached a successful termination state, which also means that it has finished executing. We can use $ \mathit{\SKIP} $ to illustrate an athlete that has crossed the finish line, or a build for a project that has passed.

Provided these two trivial processes, $ \mathit{\STOP} $ and $ \mathit{\SKIP} $, and the knowledge of what a process interface is, we can apply a handful of CSP operators to define more descriptive processes. For example, let \emph{a} be an event in the process \emph{P} interface. One can write the new process \emph{P} as $ a \then \mathit{\STOP} $, meaning that this process behaves as $ \mathit{\STOP} $ after performing \emph{a}. This operator is known as the \emph{event prefix}, and it is pronounced as ``then''.

The choice between processes can be implemented in two different ways in CSP: externally and internally. An \emph{external choice} between two processes implies the ability to perform any event that either process can engage in. Therefore, the environment has control over the outcome of such decision. On the other hand, if the process itself is the only responsible for deciding which event from its interface will be communicated, thus which process it will resolve to, then we call it an \emph{internal choice}. Note that this is essentially the source of non-determinism. To illustrate the difference among these operators, consider the following scenario: a cafeteria can operate either by letting the costumers choose between ice cream and cake for desert, or by making this choice itself, having the clients no take on what deserts they will get. In the first configuration, the choice is external to the business, whereas it is internal in the latter.

% ---
\subsection{Structured operational semantics}
% ---

Since the early 1980s there have been three complementary approaches to understanding the semantics of CSP programs. These are algebra, where we set out laws that the syntax is assumed to satisfy, behavioral models such as traces that form the basis of refinement relations and other things, and operational models, which try to understand all the actions and decisions that process implementations can make as they proceed.

The operational semantics of CSP treats the CSP language itself as a (large!) LTS. It allows us to compute the initial events of any  process, and what processes it might become after each such event. By selecting one of these actions and repeating the procedure, we can explore the state space of the process we started with. The operational semantics gives a one-state-at-a-time recipe for computing the transition system picture of any process. It is traditional to present operational semantics as a logical inference system: Plotkin’s SOS, or Structured Operational Semantics style. A process has a given action if and only if that is deducible from the rules given.

% ---
\subsection{Traces refinement}
% ---

Imagine you are interacting with a CSP process. The most basic record you might make of what happens is to write down the trace of events that occur: the sequence of communications between you (the environment) and the process. In general, a trace might be finite or infinite: finite either because the observation was terminated or because the process and environment reach a point where they cannot agree on any event; infinite when the observation goes on for ever and infinitely many events are transacted. Traces are typical of the sort of behaviors we use to build models of CSP processes: they are clearly observable by the environment interacting with the process, and each of them is the record of single interaction with the process. We will see more details of the traces model in the next chapter, but we now introduce a fundamental way in which we can specify the correctness of a CSP process. Using traces(P) to denote P’s finite traces we can write

and read this as \emph{Q trace-refines P}. In other words, every trace of \emph{Q} is a trace of \emph{P}.

% ---
\subsection{Machine-readable version of CSP}
% ---

The main purpose of CSP is to describe communicating and interacting processes. But in order to make it useful in practice we have added quite a rich language of sub-process objects: as we saw in Chap.8, CSPM contains a functional programming language to describe and manipulate things like events and process parameters.

% ---
\section{The Coq proof assistant}
% ---

Coq is a formal proof management system. It provides a formal language to write mathematical definitions, executable algorithms and theorems together with an environment for semi-interactive development of machine-checked proofs. Typical applications include the certification of properties of programming languages, the formalization of mathematics, and teaching. 

% ---
\subsection{Building proofs}
% ---

As a proof development system, Coq provides interactive proof methods, decision and semi-decision algorithms, and a tactic language for letting the user define its own proof methods. Proof development in Coq is done through a language of tactics that allows a user-guided proof process.

% ---
\subsection{The tactics language}
% ---

Ltac is the tactic language for Coq. It provides the user with a high-level “toolbox” for tactic creation, allowing one to build complex tactics by combining existing ones with constructs such as conditionals and looping.

% ---
\section{QuickChick}
% ---

QuickChick is a set of tools and techniques for combining randomized property-based testing with formal specification and proof in the Coq ecosystem.