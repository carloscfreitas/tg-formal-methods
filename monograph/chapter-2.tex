% ---
% Chapter 2
% ---
\chapter{Background}
\label{chapter:background}
% ---

Before jumping into the specifics of the implementation of CSP\textsubscript{Coq}, we need to understand some elements of the CSP language itself, such as the concrete syntax and the semantics defined in both denotational and operational models (\autoref{section:csp}). Beyond that, it is also important to provide an overview of what an interactive theorem prover is: the Coq proof assistant fundamentals such as tactics, as well as the embedded Ltac language (\autoref{section:coq}). We must also address the QuickChick property-based testing tool (\autoref{section:quickchick}), which is the Coq implementation of QuickCheck~\cite{hughes:quickcheck2000}. This chapter gives an introduction to each one of these concepts.

% ---
%TODO
\section{Communicating sequential processes}
\label{section:csp}
% ---

In 1978, Tony Hoare's \emph{Communicating Sequential Processes} \cite{hoare:csp} described a theory to help us understand concurrent systems, parallel programming and multiprocessing. More than that, it has introduced a way to decide whether a program meets its specification. This theory quickly evolved into what is known today as the CSP programming language. This language belongs to a class of notations known as process algebras, where concepts of communication and interaction are presented in an algebraic style.

Since the main goal of CSP is to provide a theory-driven framework for designing systems of interacting components an reasoning about them, we must introduce the concept of a component, or as we will be referencing it from now on, a \emph{process}. Processes are self-contained entities that once combined they can describe a system, which is yet another larger process that may itself be combined as well with other processes. The way a process communicates with the environment is through its \emph{interface}. The interface of a process is the set of all the events that the process has the potential to engage in. At last, an \emph{event} represents the atomic part of the communication itself. It is the piece of information the processes rely on to interact with one another. A process can either participate actively or passively in a communication, depending on whether it performed or suffered the action. Events may be external, meaning they appear in the process interface; indicate termination, represented by the event $ \tick $; or be internal, and therefore unknown for the environment, denoted by the event $ \tau $.

The most basic process one can define is $ \mathit{\STOP} $. Essentially, this process never interacts with the environment and its only purpose is to declare the end of an execution. In other words, it illustrates a deadlock: a state in which the process can not engage in any event or make any progress whatsoever. It could be used to describe a computer that failed booting because one of its components is damaged, or a camera that can no longer take pictures due to storage space shortage.

Another simple process is $ \mathit{\SKIP} $. It indicates that the process has reached a successful termination state, which also means that it has finished executing. We can use $ \mathit{\SKIP} $ to illustrate an athlete that has crossed the finish line, or a build for a project that has passed.

Provided these two trivial processes, $ \mathit{\STOP} $ and $ \mathit{\SKIP} $, and the knowledge of what a process interface is, we can apply a handful of CSP operators to define more descriptive processes. For example, let $ a $ be an event in the process $ P $ interface. One can write the new process $ P $ as $ a \then \mathit{\STOP} $, meaning that this process behaves as $ \mathit{\STOP} $ after performing $ a $. This operator is known as the \emph{event prefix}, and it is pronounced as ``then''.

The choice between processes can be constructed in two different ways in CSP: externally and internally. An \emph{external choice} between two processes implies the ability to perform any event that either process can engage in. Therefore, the environment has control over the outcome of such decision. On the other hand, if the process itself is the only responsible for deciding which event from its interface will be communicated, thus which process it will resolve to, then we call it an \emph{internal choice}. Note that this operator is essentially a source of non-deterministic behavior.

To illustrate the difference between these choice operators, consider the following scenario: a cafeteria may operate by either letting the costumers choose between ice cream and cake for desert, or by making this choice itself (employees decide), having the clients no take on what deserts they will get. In the first specification, the choice is external to the business and it might be described as $ ice\_cream \then \mathit{\SKIP} \extchoice cake \then \mathit{\SKIP} $, whereas it is internal in the latter, thus $ ice\_cream \then \mathit{\SKIP} \intchoice cake \then \mathit{\SKIP} $ would capture such business rule.

CSP introduces two approaches for describing a parallel execution between processes: the \emph{alphabetized parallel} and the \emph{generalized parallel}. Let $ A $ be the interface of process $ P $, and $ B $ the interface of process $ Q $. An alphabetized parallel combination of these processes is described as $ P \parallel[A][B] Q $. Events in the intersection of $ A $ and $ B $ must be simultaneously engaged in by the processes $ P $ and $ Q $. In other words, an event that appears in both process interfaces can only be communicated if the two processes are ready to perform this event. Any other event that does not match this criteria can be engaged in by its corresponding process independently. The semantics are similar for the generalized version of the parallel operator. The only change being its constructor, that takes the synchronization alphabet alone as the interface argument the processes must agree upon. Let $ C $ be the intersection of previously defined interfaces $ A $ and $ B $. The generalized parallel between process $ P $ and $ Q $ is written as $ P \parallel[C] Q $.

Both versions of the parallel operator may be used to describe a marathon where every participant is a process that runs in parallel with each other. They must all start the race at the same time, but they are not expected to cross the finish line all together. We can use the alphabetized parallel to specify the combination between two participants as $ \mathit{RUNNER1} \parallel[\{start, finish1\}][\{start, finish2\}] \mathit{RUNNER2} $, or use the generalized version of the operator instead: $ \mathit{RUNNER1} \parallel[\{start\}] \mathit{RUNNER2} $.

Another CSP operator that provides a concurrent execution of processes is the \emph{interleaving} operator. Different from the parallel operators, the interleaving represents a combination of processes that do not require any synchronization at all. The processes applied to this operation execute totally independent of each other. This might be the case of two vending machines at a supermarket. They operate completely separate from each other, receiving payments, processing changes and releasing snacks. In other words, there is no dependency regarding the communication of events between the vending machines. That being said consider the process $ \mathit{VENDING\_MACHINE} $ as $ \mathit{pay} \then \mathit{select\_snack} \then \mathit{return\_change} \then \mathit{release\_snack} $. Then, the process that specifies both machines operating together is described as $ \mathit{VENDING\_MACHINE} \interleave \mathit{VENDING\_MACHINE} $.

The last two operators we will be discussing are the \emph{sequential composition} and \emph{event hiding}. Before we continue, the reader must be aware that there are others CSP operators for combining processes apart from the ones presented in this chapter, but they will not be supported by the framework implemented in this project.

Sometimes it is necessary to pass the control over execution from one process to another, and for that we use sequential composition. It means that the first process has reached a successful termination state and now the system is ready to behave as the second process in the composition. Parents can choose to let their children play only after completing their homework. That being the case, the process $ \mathit{CHILD} $ could be modeled as $ \mathit{HOMEWORK}\!; \mathit{FUN} $, where 
\begin{align}
	&\mathit{HOMEWORK} = \mathit{choose\_subject} \then \mathit{study} \then \mathit{answer\_exercises} \then \mathit{\SKIP} \notag \\
	&\mathit{FUN} = \mathit{build\_lego} \then \mathit{watch\_cartoons} \then \mathit{play\_videogame} \then \mathit{\SKIP} \notag
\end{align}
In this example, the process $ \mathit{FUN} $ can only be executed after the process $ \mathit{HOMEWORK} $ has successfully terminated.

Last but not least, we have the event hiding operator. A system designer may choose to hide events from a process interface to prevent them from being recognized by other processes. That way, the environment can not distinguish this particular event, thus no process can engage in it. Event hiding proves to be useful when processes placed in parallel should not be allowed to synchronize on certain events. Consider, for example, that a school teacher is communicating each student individually his or her test grade. It has to be done in such way that no student gets to know other test grades besides his or her own. The process $ \mathit{TEACHER} $ may be modeled as $ \mathit{show\_grade} \then \mathit{discuss\_questions} \then \mathit{\SKIP} $, so a teacher concerned with the students privacy can be described as $ \mathit{TEACHER} \hide \{show\_grade\} $.

% ---
\subsection{Structured operational semantics}
% ---

There are three major complementary approaches for describing and reasoning about the semantics of CSP programs. These are through \emph{algebraic}, \emph{denotational} (also called \emph{behavioral}), and \emph{operational semantics}. We will be focusing in the last one, which tries to understand all the actions and decisions that process implementations can make as they proceed.

The operational semantics for CSP language describes how a valid program is interpreted as sequences of computational steps. By evaluating the initial events of a process and finding out how it will behave immediately after performing them, this approach enables us to explore the state space of any process. All we need to do is repeat this step until we have covered the transition system picture of the process we are interested in.

It is traditional to present operational semantics as a logical inference system: Plotkin’s SOS, or \emph{Structured Operational Semantics} style. A process has a given action if, and only if, that is deducible from the rules given.

We start by analyzing the process $ \mathit{\STOP} $. Since it is unable to engage in any event whatsoever, there are no inference rules for it. Then, we move forward to the next primitive process: $ \mathit{\SKIP} $. While $ \mathit{\STOP} $ has no actions of itself, $ \mathit{\SKIP} $ is able to perform a single event, which is the termination event $ \tick $. The lack of antecedents in the following rule means it is always the case that $ \mathit{\SKIP} $ may perform $ \tick $ and behave as $ \mathit{\STOP} $.

% Successful termination
\begin{prooftree}
	\AxiomC{}
	\UnaryInfC{$ \mathit{\SKIP} \trans(2)[\tick] \mathit{\STOP} $}
\end{prooftree}

The event prefix operation also spares the antecedents in its inference rule, so the conclusion is immediately deduced: if the process is initially able to perform $ a $, then after performing $ a $ it behaves like $ P $.

% Event prefix
\begin{prooftree}
	\AxiomC{}
	\UnaryInfC{$ (a \then P) \trans(2)[a] P $}
\end{prooftree}

The transition rules for external choice reflect the fact that the first external event resolves the choice in favor of the process performing the event. In addition, as we can see in the first two rules, the choice is not resolved on the occurrence of internal events. Control over resolution of the choice is external because the events of both choices are initially available. 

% External choice (tau, left side)
\begin{prooftree}
	\AxiomC{$ P \trans(2)[\tau] P' $}
	\UnaryInfC{$ P \extchoice Q \trans(2)[\tau] P' \extchoice Q $}
\end{prooftree}

% External choice (tau, right side)
\begin{prooftree}
	\AxiomC{$ Q \trans(2)[\tau] Q' $}
	\UnaryInfC{$ P \extchoice Q \trans(2)[\tau] P \extchoice Q' $}
\end{prooftree}

% External choice (!tau, left side)
\begin{prooftree}
	\AxiomC{$ P \trans(2)[a] P' $}
	\RightLabel{\quad ($ a \neq \tau $)}
	\UnaryInfC{$ P \extchoice Q \trans(2)[a] P' $}
\end{prooftree}

% External choice (!tau, right side)
\begin{prooftree}
	\AxiomC{$ Q \trans(2)[a] Q' $}
	\RightLabel{\quad ($ a \neq \tau $)}
	\UnaryInfC{$ P \extchoice Q \trans(2)[a] Q' $}
\end{prooftree}

The internal choice is an operation that guarantees the process to behave as either of its components on any execution. This state change happens ``silently'', thus this transition is followed by the communication of internal event $ \tau $, as we can see in the inference rules for this operation.

% Internal choice (left side)
\begin{prooftree}
	\AxiomC{}
	\UnaryInfC{$ P \intchoice Q \trans(2)[\tau] P $}
\end{prooftree}

% Internal choice (right side)
\begin{prooftree}
	\AxiomC{}
	\UnaryInfC{$ P \intchoice Q \trans(2)[\tau] Q $}
\end{prooftree}

We can separate the rules for the alphabetized parallel into two categories: one that describes the independent execution of each process, and other defining the synchronized step performed at once by the components. The first two inference rules capture the ability of both sides performing events that are not in the common interface, thus executing them independently. The third rule dictates the joint step, where both processes are able to perform the event, so they communicate it at the same time.

% Alphabetized parallel (left side)
\begin{prooftree}
	\AxiomC{$ P \trans(2)[\mu] P' $}
	\RightLabel{\quad ($ \mu \in (A \cup \{\, \tau \,\} \setminus B) $)}
	\UnaryInfC{$ P \parallel[A][B] Q \trans(2)[\mu] P' \parallel[A][B] Q $}
\end{prooftree}

% Alphabetized parallel (right side)
\begin{prooftree}
	\AxiomC{$ Q \trans(2)[\mu] Q' $}
	\RightLabel{\quad ($ \mu \in (B \cup \{\, \tau \,\} \setminus A) $)}
	\UnaryInfC{$ P \parallel[A][B] Q \trans(2)[\mu] P \parallel[A][B] Q' $}
\end{prooftree}

% Alphabetized parallel (sync)
\begin{prooftree}
	\AxiomC{$ P \trans(2)[a] P' $}
	\AxiomC{$ Q \trans(2)[a] Q' $}
	\RightLabel{\quad ($ a \in A^{\tick} \cap B^{\tick} $)}
	\BinaryInfC{$ P \parallel[A][B] Q \trans(2)[a] P' \parallel[A][B] Q' $}
\end{prooftree}

The transition rules for the generalized parallel are very similar to the ones for the previous operation. The main difference lies in the side condition, since this version of parallelism is only interested in the interface alphabet. The same rule categories for the alphabetized parallel apply to this operation.

% Generalized parallel (left side)
\begin{prooftree}
	\AxiomC{$ P \trans(2)[\mu] P' $}
	\RightLabel{\quad ($ \mu \notin A^{\tick} $)}
	\UnaryInfC{$ P \parallel[A] Q \trans(2)[\mu] P' \parallel[A] Q $}
\end{prooftree}

% Generalized parallel (right side)
\begin{prooftree}
	\AxiomC{$ Q \trans(2)[\mu] Q' $}
	\RightLabel{\quad ($ \mu \notin A^{\tick} $)}
	\UnaryInfC{$ P \parallel[A] Q \trans(2)[\mu] P \parallel[A] Q' $}
\end{prooftree}

% Generalized parallel (sync)
\begin{prooftree}
	\AxiomC{$ P \trans(2)[a] P' $}
	\AxiomC{$ Q \trans(2)[a] Q' $}
	\RightLabel{\quad ($ a \in A^{\tick} $)}
	\BinaryInfC{$ P \parallel[A] Q \trans(2)[a] P' \parallel[A] Q' $}
\end{prooftree}

The interleave operation describes a parallel execution between processes that do not synchronize in any event except termination $ \tick $. In other words, this operation is a particular case of the generalized parallelism, where the interface alphabet is empty, thus the event $ \tick $ being the only event that can be performed simultaneously by the components.

% Interleave (left side)
\begin{prooftree}
	\AxiomC{$ P \trans(2)[\mu] P' $}
	\RightLabel{\quad ($ \mu \neq \tick $)}
	\UnaryInfC{$ P \interleave Q \trans(2)[\mu] P' \interleave Q $}
\end{prooftree}

% Interleave (right side)
\begin{prooftree}
	\AxiomC{$ Q \trans(2)[\mu] Q' $}
	\RightLabel{\quad ($ \mu \neq \tick $)}
	\UnaryInfC{$ P \interleave Q \trans(2)[\mu] P \interleave Q' $}
\end{prooftree}

% Interleave (tick)
\begin{prooftree}
	\AxiomC{$ P \trans(2)[\tick] P' $}
	\AxiomC{$ Q \trans(2)[\tick] Q' $}
	\BinaryInfC{$ P \interleave Q \trans(2)[\tick] P' \interleave Q' $}
\end{prooftree}

As we already know, the hiding operator removes all events in a given alphabet from the process interface, preventing other processes to engage in them. The process to which the event hiding is applied can then behave just like it would without the operator, except the events in the given alphabet are made internal and then renamed to $ \tau $. Such behavior is capture by the inference rules:

% Event hiding
\begin{prooftree}
	\AxiomC{$ P \trans(2)[a] P' $}
	\RightLabel{\quad ($ a \in A $)}
	\UnaryInfC{$ P \hide A \trans(2)[\tau] P' \hide A $}
\end{prooftree}

% Event hiding
\begin{prooftree}
	\AxiomC{$ P \trans(2)[\mu] P' $}
	\RightLabel{\quad ($ \mu \notin A $)}
	\UnaryInfC{$ P \hide A \trans(2)[\mu] P' \hide A $}
\end{prooftree}

The last operational rules we need to discuss are for the sequential composition operator. Initially, this combination behaves as the process to the left of the operator until it terminates. Then, the execution control is granted to the other process in the composition. The control handover is represented by the communication of the internal event $ \tau $, as we can see in the second rule:

% Sequential composition (!tick)
\begin{prooftree}
	\AxiomC{$ P \trans(2)[a] P' $}
	\RightLabel{\quad ($ a \neq \tick $)}
	\UnaryInfC{$ P\!; Q \trans(2)[a] P'\!; Q $}
\end{prooftree}

% Sequential composition (tick)
\begin{prooftree}
	\AxiomC{$ P \trans(2)[\tick] P' $}
	\UnaryInfC{$ P\!; Q \trans(2)[\tau] Q $}
\end{prooftree}


% ---
\subsection{Traces refinement}
% ---

A pretty reasonable way for gathering information from a process interacting with the environment is by keeping track of the events this process engages in. This sequence of communication between process and environment, presented in a chronological order, is what we call a \emph{trace}. Traces can either be finite or infinite, and it depends on the observation span and the nature of the process itself.

Because this record is easily observed by the environment and it represents a single interaction, it is often used to build models of CSP processes. As a matter of fact, there is one named after it: the $ \mathit{\traces} $ model, represented by the symbol $ \tmodel $. It defines the meaning of a process expression as the set of sequences of events (traces) that the process can be observed to perform. This model is one of the three major denotational models of CSP, the other ones being the \emph{stable} $ \mathit{\failures} \ \fmodel $ and the $ \mathit{\failures \mhyphen \divergences} $ model $ \nmodel $.

The notion of refinement is a particularly useful concept for specifying the correctness of a CSP process. If we can establish a relation between components of a system which captures the fact that one satisfies at least the same conditions as another, then we may replace a worse component by a better one without degrading the properties of the system.

\begin{definition}{\textbf{(Traces Refinement)}}
	Let $ P $ and $ Q $ be two CSP processes, and $ \mathit{\traces} $ be a function that yields the set of all possible traces of a given CSP process, we say that $ Q $ trace-refines $ P $ if, and only if, every trace of $ Q $ is also a trace of $ P $:
	\[  P \refinedby[\tmodel] Q \iff \mathit{\traces}(Q) \subseteq \mathit{\traces}(P) \]
\end{definition}

If we consider $ P $ to be a specification which determines possible safe states of a system, then we can think of $ P \refinedby[\tmodel] Q $ as saying that $ Q $ is a safe implementation: no wrong events will be allowed. 

% ---
\subsection{Machine-readable version of CSP}
% ---

In the beginning, CSP was typically used as a blackboard language. In other words, it was conceived to describe communicating and interacting processes for a human audience. Theories such as CSP have a higher chance of acceptance among the industry and academy (e.g. for teaching purposes) when they have tool support available. For that reason, the need of a notation that could actually be used with tools emerged.

The machine-readable CSP, usually denoted as \CSPM{}, not only provides a notation for tools such as FDR model-checker to be build upon but also extends the existing theory by using a functional programming language to describe and manipulate things like events and process parameters.

The \autoref{tab:csp-cspm} shows for every CSP process constructor discussed in \autoref{section:csp} the corresponding ASCII representation according to \CSPM{} language.

\begin{table}[htb]
	\begin{center}
		\caption[The ASCII representation of CSP]{The ASCII representation of CSP.}
		\label{tab:csp-cspm}
		\begin{tabular}{ |l|c|c| }
			\hline
			Constructor & Syntax & ASCII form \\
			\hline
			Stop & $ \mathit{\STOP} $ & STOP \\ [0.5ex]
			Skip & $ \mathit{\SKIP} $ & SKIP \\ [0.5ex]
			Event prefix & $ e \then P $ & e -> P \\  [0.5ex]
			External choice & $ P \extchoice Q $ & P [] Q \\  [0.5ex]
			Internal choice & $ P \intchoice Q $ & P |$ \sim $| Q \\ [0.5ex]
			Alphabetized parallel & $ P \parallel[A][B] Q $ & P [A || B] Q \\ [0.5ex]
			Generalized parallel & $ P \parallel[A] Q $ & P [| A |] Q \\ [0.5ex]
			Interleave & $ P \interleave Q $ & P ||| Q \\ [0.5ex]
			Sequential composition & $ P ; Q $ & P ; Q \\ [0.5ex]
			Event hiding & $ P \hide A $ & P \textbackslash \ A \\ [0.5ex]
			\hline
		\end{tabular}
	\end{center}
\end{table}

% ---
\section{The Coq proof assistant}
\label{section:coq}
% ---

A proof assistant is a software for helping construct proofs of logical propositions. Essentially, it is a hybrid tool that automates the more routine aspects of building proofs while relying on human intervention for more complex steps. There is a variety of proof assistants including Isabelle, Agda, ATS, Idris and Coq, among others. This work is based around the Coq proof assistant. 

Coq can be viewed as a combination of a functional programming language plus a set of tools for stating and proving logical assertions. Moreover, the Coq environment provides high-level facilities for proof development, including a large library of common definitions and lemmas, powerful tactics for constructing complex proofs semi-automatically, and a special-purpose programming language for defining new proof-automation tactics for specific situations.

Coq's native functional programming language is called \emph{Gallina}. Before we discuss about the proof development aspect of this interactive theorem prover, we need to introduce the most essential elements we may find in a Gallina program. Consider the following definition of natural numbers in Coq:

\begin{coqdoccode}
	\coqdocnoindent
	\coqdockw{Inductive} \coqdocvar{nat} : \coqdockw{Type} :=\coqdoceol
	\coqdocindent{1.00em}
	\ensuremath{|} \coqdocvar{O}\coqdoceol
	\coqdocindent{1.00em}
	\ensuremath{|} \coqdocvar{S} (\coqdocvar{n} : \coqdocvar{nat}).\coqdoceol
\end{coqdoccode}

This declaration tells Coq that we are defining a \emph{type}. The capital-letter $ O $ constructor represents zero. When the $ S $ constructor is applied to the representation of the natural number $ n $, the result is the representation of $ n+1 $, where $ S $ stands for "successor". An $ \coqdockw{Inductive} $ definition carves out a subset of the whole space of constructor expressions and gives it a name, in this case, $ nat $. 

Having defined $ nat $, we can write functions that operate on natural numbers, such as the predecessor function:

\begin{coqdoccode}
	\coqdocnoindent
	\coqdockw{Definition} \coqdocvar{pred} (\coqdocvar{n} : \coqdocvar{nat}) : \coqdocvar{nat} :=\coqdoceol
	\coqdocindent{1.00em}
	\coqdockw{match} \coqdocvar{n} \coqdockw{with}\coqdoceol
	\coqdocindent{2.00em}
	\ensuremath{|} \coqdocvar{O} \ensuremath{\Rightarrow} \coqdocvar{O}\coqdoceol
	\coqdocindent{2.00em}
	\ensuremath{|} \coqdocvar{S} \coqdocvar{n'} \ensuremath{\Rightarrow} \coqdocvar{n'}\coqdoceol
	\coqdocindent{1.00em}
	\coqdockw{end}.\coqdoceol
\end{coqdoccode}

Note that we do not need recursion to define the predecessor function, but simple pattern matching is not enough for more interesting computations involving natural numbers. For example, to check that a number $ n $ is even, we may need to recursively check whether $ n-2 $ is even. In order to do that, we use the keyword $ \coqdockw{Fixpoint} $ instead of $ \coqdockw{Definition} $: 

\begin{coqdoccode}
	\coqdocnoindent
	\coqdockw{Fixpoint} \coqdocvar{evenb} (\coqdocvar{n}:\coqdocvar{nat}) : \coqdocvar{bool} :=\coqdoceol
	\coqdocindent{1.00em}
	\coqdockw{match} \coqdocvar{n} \coqdockw{with}\coqdoceol
	\coqdocindent{2.00em}
	\ensuremath{|} \coqdocvar{O} \ensuremath{\Rightarrow} \coqdocvar{true}\coqdoceol
	\coqdocindent{2.00em}
	\ensuremath{|} \coqdocvar{S} \coqdocvar{O} \ensuremath{\Rightarrow} \coqdocvar{false}\coqdoceol
	\coqdocindent{2.00em}
	\ensuremath{|} \coqdocvar{S} (\coqdocvar{S} \coqdocvar{n'}) \ensuremath{\Rightarrow} \coqdocvar{evenb} \coqdocvar{n'}\coqdoceol
	\coqdocindent{1.00em}
	\coqdockw{end}.\coqdoceol
\end{coqdoccode}

Yet another way for defining evenness is through \emph{inductive declaration}. Consider the following two rules: \emph{the number $ 0 $ is even}, and \emph{if $ n $ is even, then $ S \ (S \ n) $ is even}. Lets call the first rule $ ev\_0 $ and then the second $ ev\_SS $. Using $ ev $ for the name of evenness property, we can write the following inference rules:

\begin{prooftree}
	\AxiomC{}
	\RightLabel{\quad ($ ev\_0 $)}
	\UnaryInfC{$ ev \ 0 $}
\end{prooftree}

\begin{prooftree}
	\AxiomC{$ ev \ n $}
	\RightLabel{\quad ($ ev\_SS $)}
	\UnaryInfC{$ ev \ (S \ (S \ n)) $}
\end{prooftree}

Now, we can translate these rules into a formal Coq definition. Each constructor in this definition corresponds to an inference rule:

\begin{coqdoccode}
	\coqdocnoindent
	\coqdockw{Inductive} \coqdocvar{ev} : \coqdocvar{nat} \ensuremath{\rightarrow} \coqdockw{Prop} :=\coqdoceol
	\coqdocindent{1.00em}
	\ensuremath{|} \coqdocvar{ev\_0} : \coqdocvar{ev} 0\coqdoceol
	\coqdocindent{1.00em}
	\ensuremath{|} \coqdocvar{ev\_SS} (\coqdocvar{n} : \coqdocvar{nat}) (\coqdocvar{H} : \coqdocvar{ev} \coqdocvar{n}) : \coqdocvar{ev} (\coqdocvar{S} (\coqdocvar{S} \coqdocvar{n})).\coqdoceol
\end{coqdoccode}

This definition is different from previous use of $ \coqdockw{Inductive} $. We are defining a function from $ nat $ to $ Prop $, in other words, a property of numbers. The type of each constructor must be specified explicitly (after a colon), and each constructor's type must have the form $ ev \ n $ for some natural number $ n $.

% ---
\subsection{Building proofs}
% ---

As a proof development system, Coq provides interactive proof methods, decision and semi-decision algorithms, and a tactic language for letting the user define its own proof methods. Proof development in Coq is done through a language of tactics that allows a user-guided proof process.

For instance, consider the following theorem on natural numbers: for all natural $ n $, if $ n $ is even, then the predecessor of the predecessor of $ n $ is also even. This theorem can be proved in Coq using the following commands:

\begin{coqdoccode}
	\coqdocnoindent
	\coqdockw{Theorem} \coqdocvar{ev\_minus2} : \coqdockw{\ensuremath{\forall}} \coqdocvar{n}:\coqdocvar{nat},\coqdoceol
	\coqdocindent{1.00em}
	\coqdocvar{ev} \coqdocvar{n} \ensuremath{\rightarrow} \coqdocvar{ev} (\coqdocvar{pred} (\coqdocvar{pred} \coqdocvar{n})).\coqdoceol
	\coqdocnoindent
	\coqdockw{Proof}.\coqdoceol
	\coqdocindent{1.00em}
	\coqdoctac{intros}.\coqdoceol
	\coqdocindent{1.00em}
	\coqdoctac{destruct} \coqdocvar{H}.\coqdoceol
	\coqdocindent{1.00em}
	- \coqdoctac{simpl}. \coqdoctac{apply} \coqdocvar{ev\_0}.\coqdoceol
	\coqdocindent{1.00em}
	- \coqdoctac{simpl}. \coqdoctac{apply} \coqdocvar{H}.\coqdoceol
	\coqdocnoindent
	\coqdockw{Qed}.\coqdoceol
\end{coqdoccode}

The proof editing mode in Coq is entered whenever asserting a statement. The command \coqdockw{Theorem} does so by allowing us to give the theorem a name and providing it with the proposition we want to prove. Additionally, the commands \coqdockw{Proof} and \coqdockw{Qed} delimit, respectively, the beginning and end of the sequence of tactic commands.

The keywords \coqdoctac{intros}, \coqdoctac{simpl}, \coqdoctac{destruct} and \coqdoctac{apply} are examples of tactics. A tactic is a command that is used to guide the process of checking some claim we are making. Note that we have added the quantifier \coqdockw{\ensuremath{\forall}} \coqdocvar{n}:\coqdocvar{nat}, so that our theorem talks about all natural numbers $ n $. Therefore, the tactic \coqdoctac{intros} moves $ n $ from the quantifier in the goal to a context of current assumptions, along with a new \emph{hypothesis} that consists of the antecedent of the implication (\emph{modus ponens}).

Proving a statement by case analysis is possible via the tactic \coqdoctac{destruct}. In our example, this tactic is responsible for generating, from the hypothesis, two sub-goals based on the inductive definition of evenness we have provided: one where $ n = 0 $ and the other where $ n = S \ (S \ n) $. Thus, we must prove them separately in order to get Coq to accept the theorem.

The tactic \coqdoctac{simpl} is used in situations where we want to evaluate a compound expression, eventually reducing it to a simplified, easier-to-understand term. It facilitates our decisions in a proof development by resolving all the computations that can be done in a given state of the goal.

We then proceed to use the parameterized tactic \coqdoctac{apply}, passing the term we find useful for proving each sub-goal. In the first branch, where our goal is to prove $ ev \ 0 $, we apply the first rule of our inductive definition, $ ev\_0 $, which concludes this sub-proof. In the second branch, we have $ ev \ n $ as goal, which is already an assumption of ours (introduced to the proof context by the command \coqdoctac{intros}). This also concludes the second sub-proof, thus proving the entire statement (\coqdockw{Theorem} \coqdocvar{ev\_minus2}).

There are many other tactics and variations of them that can be used when proving a statement. Apart from the ones we have already discussed in this subsection, other commonly used tactic commands are \coqdoctac{reflexivity}, \coqdoctac{rewrite}, \coqdoctac{unfold}, \coqdoctac{induction}, \coqdoctac{inversion} and \coqdoctac{contradiction}. Further explanation on theses tactics will be provided as needed throughout the \autoref{chapter:csp_coq}.

% ---
%TODO
\subsection{The tactics language}
% ---

Ltac is the tactic language for Coq. It provides the user with a high-level ``toolbox''  for tactic creation, allowing one to build complex tactics by combining existing ones with constructs such as conditionals, looping, backtracking, and error catching.

Imagine we want to prove that the number 4 does not appear in the list of consecutive natural numbers ranging from 0 to 3. By using the keyword \coqdockw{Example}, we can assert this statement in Coq and develop our proof:

\begin{coqdoccode}
	\coqdocnoindent
	\coqdockw{Example} \coqdocvar{elem\_not\_in\_list} : \ensuremath{\lnot} (\coqdocvar{In} 4 [0 ; 1 ; 2 ; 3]).\coqdoceol
	\coqdocnoindent
	\coqdockw{Proof}.\coqdoceol
	\coqdocindent{1.00em}
	\coqdoctac{unfold} \coqdocvar{not}. \coqdoctac{simpl}. \coqdoctac{intros}.\coqdoceol
	\coqdocindent{1.00em}
	\coqdoctac{destruct} \coqdocvar{H}.\coqdoceol
	\coqdocindent{1.00em}
	- \coqdoctac{inversion} \coqdocvar{H}.\coqdoceol
	\coqdocindent{1.00em}
	- \coqdoctac{destruct} \coqdocvar{H}.\coqdoceol
	\coqdocindent{2.00em}
	\ensuremath{\times} \coqdoctac{inversion} \coqdocvar{H}.\coqdoceol
	\coqdocindent{2.00em}
	\ensuremath{\times} \coqdoctac{destruct} \coqdocvar{H}.\coqdoceol
	\coqdocindent{3.00em}
	+ \coqdoctac{inversion} \coqdocvar{H}.\coqdoceol
	\coqdocindent{3.00em}
	+ \coqdoctac{destruct} \coqdocvar{H}.\coqdoceol
	\coqdocindent{4.00em}
	\{ \coqdoctac{inversion} \coqdocvar{H}. \}\coqdoceol
	\coqdocindent{4.00em}
	\{ \coqdoctac{contradiction}. \}\coqdoceol
	\coqdocnoindent
	\coqdockw{Qed}.\coqdoceol
\end{coqdoccode}

\noindent See the proof for Example elem\_not\_in\_list in table \ref{Proofs:coq-example:Example:elem-not-in-list}

\begin{longtable}{| S | P |}
	\caption{Proof of Example elem\_not\_in\_list}\\
	\hline
	\coqpsvstephdr & \coqpsvsithdr\\
	\hline
	\endfirsthead
	
	\caption{Proof of Example elem\_not\_in\_list continued}\\
	\hline
	\coqpsvstephdr & \coqpsvsithdr\\
	\hline
	\endhead
	
	\multicolumn{2}{r}{Continuing proof of Example elem\_not\_in\_list on the next page}\\
	\endfoot
	\hline
	\multicolumn{2}{r}{End of proof of Example elem\_not\_in\_list}\\
	\endlastfoot
	
	\multirow{2}{=}{$Proof.$} & \fracrule\linebreak
	$ \lnot $ $ In $ $ 4 $ $ [0; $ $ 1; $ $ 2; $ $ 3]$\\
	
	\hline
	\multirow{2}{=}{$unfold $ $ not.$} & \fracrule\linebreak
	$In $ $ 4 $ $ [0; $ $ 1; $ $ 2; $ $ 3] $ $ \rightarrow $ $ False$\\
	
	\hline
	\multirow{2}{=}{$simpl.$} & \fracrule\linebreak
	$0 $ $ = $ $ 4 $ $ \lor $ $ 1 $ $ = $ $ 4 $ $ \lor $ $ 2 $ $ = $ $ 4 $ $ \lor $ $ 3 $ $ = $ $ 4 $ $ \lor $ $ False $ $ \rightarrow $ $ False$\\
	
	\hline
	\multirow{3}{=}{$intros.$} & $H$$ $ $ : $ $ 0 $ $ = $ $ 4 $ $ \lor $ $ 1 $ $ = $ $ 4 $ $ \lor $ $ 2 $ $ = $ $ 4 $ $ \lor $ $ 3 $ $ = $ $ 4 $ $ \lor $ $ False$\linebreak
	\fracrule\linebreak
	$False$\\
	
	\hline
	\multirow{8}{=}{$destruct $ $ H.$} & $H$$ $ $ : $ $ 0 $ $ = $ $ 4$\linebreak
	\fracrule\linebreak
	$False$\\
	\cline{2-2}
	& $H$$ $ $ : $ $ 1 $ $ = $ $ 4 $ $ \lor $ $ 2 $ $ = $ $ 4 $ $ \lor $ $ 3 $ $ = $ $ 4 $ $ \lor $ $ False$\linebreak
	\fracrule\linebreak
	$False$\\
	
	\hline
	\multirow{3}{=}{$-_{1/2}$} & $H$$ $ $ : $ $ 0 $ $ = $ $ 4$\linebreak
	\fracrule\linebreak
	$False$\\
	
	\hline
	$inversion $ $ H.$ & $-_{1/2}$ completed \\
	\hline
	\multirow{3}{=}{$-_{2/2}$} & $H$$ $ $ : $ $ 1 $ $ = $ $ 4 $ $ \lor $ $ 2 $ $ = $ $ 4 $ $ \lor $ $ 3 $ $ = $ $ 4 $ $ \lor $ $ False$\linebreak
	\fracrule\linebreak
	$False$\\
	
	\hline
	\multirow{7}{=}{$destruct $ $ H.$} & $H$$ $ $ : $ $ 1 $ $ = $ $ 4$\linebreak
	\fracrule\linebreak
	$False$\\
	\cline{2-2}
	& $H$$ $ $ : $ $ 2 $ $ = $ $ 4 $ $ \lor $ $ 3 $ $ = $ $ 4 $ $ \lor $ $ False$\linebreak
	\fracrule\linebreak
	$False$\\
	
	\hline
	\multirow{3}{=}{$*_{1/2}$} & $H$$ $ $ : $ $ 1 $ $ = $ $ 4$\linebreak
	\fracrule\linebreak
	$False$\\
	
	\hline
	$inversion $ $ H.$ & $*_{1/2}$ completed \\
	\hline
	\multirow{4}{=}{$*_{2/2}$} & $H$$ $ $ : $ $ 2 $ $ = $ $ 4 $ $ \lor $ $ 3 $ $ = $ $ 4 $ $ \lor $ $ False$\linebreak
	\fracrule\linebreak
	$False$\\
	
	\hline
	\multirow{6}{=}{$destruct $ $ H.$} & $H$$ $ $ : $ $ 2 $ $ = $ $ 4$\linebreak
	\fracrule\linebreak
	$False$\\
	\cline{2-2}
	& $H$$ $ $ : $ $ 3 $ $ = $ $ 4 $ $ \lor $ $ False$\linebreak
	\fracrule\linebreak
	$False$\\
	
	\hline
	\multirow{3}{=}{$+_{1/2}$} & $H$$ $ $ : $ $ 2 $ $ = $ $ 4$\linebreak
	\fracrule\linebreak
	$False$\\
	
	\hline
	$inversion $ $ H.$ & $+_{1/2}$ completed \\
	\hline
	\multirow{3}{=}{$+_{2/2}$} & $H$$ $ $ : $ $ 3 $ $ = $ $ 4 $ $ \lor $ $ False$\linebreak
	\fracrule\linebreak
	$False$\\
	
	\hline
	\multirow{6}{=}{$destruct $ $ H.$} & $H$$ $ $ : $ $ 3 $ $ = $ $ 4$\linebreak
	\fracrule\linebreak
	$False$\\
	\cline{2-2}
	& $H$$ $ $ : $ $ False$\linebreak
	\fracrule\linebreak
	$False$\\
	
	\hline
	\multirow{3}{=}{$\{$} & $H$$ $ $ : $ $ 3 $ $ = $ $ 4$\linebreak
	\fracrule\linebreak
	$False$\\
	
	\hline
	$inversion $ $ H.$ & $Subproof$ completed \\
	\hline
	\multirow{3}{=}{$\}$} & $H$$ $ $ : $ $ False$\linebreak
	\fracrule\linebreak
	$False$\\
	
	\hline
	\multirow{3}{=}{$\{$} & $H$$ $ $ : $ $ False$\linebreak
	\fracrule\linebreak
	$False$\\
	
	\hline
	$contradiction.$ & $Subproof$ completed, proof completed by Qed \\
	\hline
	$\}$ & $+_{2/2}$ completed, proof completed by Qed \label{Proofs:coq-example:Example:elem-not-in-list} \\
	\hline
	
\end{longtable}

The first tactic unfolds the definition of \ensuremath{\lnot} (not) in the goal, replacing our initial statement by \coqdocvar{In} 4 [0 ; 1 ; 2 ; 3] \ensuremath{\rightarrow} \coqdocvar{False}. The second tactic reduces the new goal by computing the function \coqdocvar{In}, which leaves us with the disjunctions $ 0 = 4 \lor 1 = 4 \lor 2 = 4 \lor 3 = 4 \lor \coqdocvar{False} $ in the antecedent of the implication. Then, the tactic \coqdoctac{intros} moves this antecedent to the proof context, introducing a new hypothesis \coqdocvar{H} and leaving the literal \coqdocvar{False} as goal.

From this point on, the proof develops a pattern: we perform a destruction followed by an inversion of the hypothesis until we can end the proof by contradiction. The recurrence of the tactic \coqdoctac{destruct} lets us focus in one equality from the disjunction at a time: first the hypothesis becomes $ 0 = 4 $, then $ 1 = 4 $ and so on. The tactic \coqdoctac{inversion} finishes each sub proof created by the previous command by deriving all the necessary conditions that should hold for the assumption to be proved. In this case, since none of these equalities is true, there is no condition that satisfies the proposition, thus proving our goal.

Since this pattern in the sequence of tactics is now exposed, we can define a tactic macro for proving propositions of the format ``not in'' using the keyword \coqdockw{Ltac}:

\begin{coqdoccode}
	\coqdocnoindent
	\coqdockw{Ltac} \coqdocvar{solve\_not\_in} := \coqdoctac{unfold} \coqdocvar{not};\coqdoceol
	\coqdocindent{1.00em}
	\coqdockw{let} \coqdocvar{H} := \coqdoctac{fresh} "H" \coqdoctac{in} (\coqdoceol
	\coqdocindent{2.00em}
	\coqdoctac{intros} \coqdocvar{H}; \coqdoctac{repeat} (\coqdoctac{contradiction} + (\coqdoctac{destruct} \coqdocvar{H}; [> \coqdoctac{inversion} \coqdocvar{H} \ensuremath{|} ]))\coqdoceol
	\coqdocindent{1.00em}
	).\coqdoceol
	\coqdocemptyline
	\coqdocnoindent
	\coqdockw{Example} \coqdocvar{elem\_not\_in\_list'} : \ensuremath{\lnot} (\coqdocvar{In} 4 [0 ; 1 ; 2 ; 3]).\coqdoceol
	\coqdocnoindent
	\coqdockw{Proof}. \coqdocvar{solve\_not\_in}. \coqdockw{Qed}.\coqdoceol
\end{coqdoccode}

The new tactic \coqdocvar{solve\_not\_in} works by first unfolding the function \coqdocvar{not}, therefore deriving an implication, then moving the premise to the context of assumption, and finally repeating the following steps until the proof is finished: try to finish the proof by searching for a \emph{contradiction} in the assumptions, if it fails, \emph{destruct} the hypothesis and apply \emph{inversion} to it in order to prove the first sub-goal yielded by the previous tactic.

\noindent See the proof for Example elem\_not\_in\_list' in table \ref{Proofs:coq-example:Example:elem-not-in-list'}

\begin{longtable}{| S | P |}
	\caption{Proof of Example elem\_not\_in\_list'}\\
	\hline
	\coqpsvstephdr & \coqpsvsithdr\\
	\hline
	\endfirsthead
	
	\caption{Proof of Example elem\_not\_in\_list' continued}\\
	\hline
	\coqpsvstephdr & \coqpsvsithdr\\
	\hline
	\endhead
	
	\multicolumn{2}{r}{Continuing proof of Example elem\_not\_in\_list' on the next page}\\
	\endfoot
	\hline
	\multicolumn{2}{r}{End of proof of Example elem\_not\_in\_list'}\\
	\endlastfoot
	
	\multirow{2}{=}{$Proof.$} & \fracrule\linebreak
	${\char'176} $ $ In $ $ 4 $ $ [0; $ $ 1; $ $ 2; $ $ 3]$\\
	
	\hline
	$solve{\char`\_}not{\char`\_}in.$ & Proof finished by Qed \label{Proofs:coq-example:Example:elem-not-in-list'} \\
	\hline
\end{longtable}

% ---
%TODO
\section{QuickChick}
\label{section:quickchick}
% ---

QuickChick is a set of tools and techniques for combining randomized property-based testing with formal specification and proof in the Coq ecosystem. It is the equivalent of Haskell's QuickCheck for Coq proof assistant.

There are four basic elements in property-based random testing: an \emph{executable property} such as for deciding whether a number is even, \emph{generators} for random inputs to the property, \emph{printers} for converting data structures like numbers to strings when reporting counterexamples, and \emph{shrinkers}, which are used to search for minimal counterexamples when errors occur.
